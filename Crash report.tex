%%%%%%%%%%%%%%%%%%%%%%%%%%%%%%%%%%%%%%%%%%%%%%%%%%%%%%%%%%%%%%%%%%%%%%
% How to use shareLaTeX: 
%
% You edit the source code here on the left, and the preview on the
% right shows you the result within a few seconds.
%
% Bookmark this page and share the URL with your co-authors. They can
% edit at the same time!
%
% You can upload figures, bibliographies, custom classes and
% styles using the files menu.
%
% If you're new to LaTeX, the wikibook is a great place to start:
% http://en.wikibooks.org/wiki/LaTeX
%
%%%%%%%%%%%%%%%%%%%%%%%%%%%%%%%%%%%%%%%%%%%%%%%%%%%%%%%%%%%%%%%%%%%%%%
%
% Template for PLoS
% Version 1.0 January 2009
%
% Edit the title below to update the display in My Documents
%\title{PLoS Journal Article}

\documentclass[10pt]{article}

% amsmath package, useful for mathematical formulas
\usepackage{amsmath}
% amssymb package, useful for mathematical symbols
\usepackage{amssymb}

% graphicx package, useful for including eps and pdf graphics
% include graphics with the command \includegraphics
\usepackage{graphicx}

% cite package, to clean up citations in the main text. Do not remove.
\usepackage{cite}

\usepackage{color} 

\usepackage[utf8]{inputenc}

\usepackage[none]{hyphenat}


\usepackage{titlesec}
\titlespacing\section{0pt}{12pt plus 4pt minus 2pt}{0pt plus 2pt minus 2pt}


% Use doublespacing - comment out for single spacing
\usepackage{setspace} 

% Text layout
\topmargin 0.0cm
\oddsidemargin 0.5cm
\evensidemargin 0.5cm
\textwidth 18cm 
\textheight 25cm

% Bold the 'Figure #' in the caption and separate it with a period
% Captions will be left justified
\usepackage[labelfont=bf,labelsep=period,justification=raggedright]{caption}

% Use the PLoS provided bibtex style
\bibliographystyle{plos2009}

% Remove brackets from numbering in List of References
\makeatletter
\renewcommand{\@biblabel}[1]{\quad#1.}
\makeatother


% Remove page numbers
%\pagenumbering{gobble}



% Change margins
\usepackage[top=2.5cm, bottom=2.5cm, left=2.5cm, right=2.5cm]{geometry}

% Leave date blank
\date{}

\pagestyle{myheadings}
%% ** EDIT HERE **


%% ** EDIT HERE **
%% PLEASE INCLUDE ALL MACROS BELOW

%% END MACROS SECTION

\begin{document}

% Title must be 150 characters or less

\begin{center}
{\Large
\textbf{CRASH - Comfort, Reliability and Self Handling}
}
\end{center}


\tableofcontents
\newpage

% Projektets bakgrund
\section{Bakgrund}
\sloppy
\noindent Lorem ipsum dolor sit amet, consectetur adipisicing elit, sed do eiusmod tempor incididunt ut labore et dolore magna aliqua. Ut enim ad minim veniam, quis nostrud exercitation ullamco laboris nisi ut aliquip ex ea commodo consequat. Duis aute irure dolor in reprehenderit in voluptate velit esse cillum dolore eu fugiat nulla pariatur. Excepteur sint occaecat cupidatat non proident, sunt in culpa qui officia deserunt mollit anim id est laborum.

% Projektets mål   
\section{Mål}
\sloppy
\noindent Lorem ipsum dolor sit amet, consectetur adipisicing elit, sed do eiusmod tempor incididunt ut labore et dolore magna aliqua. Ut enim ad minim veniam, quis nostrud exercitation ullamco laboris nisi ut aliquip ex ea commodo consequat. Duis aute irure dolor in reprehenderit in voluptate velit esse cillum dolore eu fugiat nulla pariatur. Excepteur sint occaecat cupidatat non proident, sunt in culpa qui officia deserunt mollit anim id est laborum.

% Projektets funktionalitet
\section{Funktionalitet}
\sloppy
\noindent Lorem ipsum dolor sit amet: 
\begin{itemize}
	\setlength\itemsep{0.1em}
	\item Lorem ipsum dolor sit amet
	\item Lorem ipsum dolor sit amet
	\item Lorem ipsum dolor sit amet
	\item Lorem ipsum dolor sit amet
	\item Lorem ipsum dolor sit amet
	\item Lorem ipsum dolor sit amet
	\item Lorem ipsum dolor sit amet
	\item Lorem ipsum dolor sit amet
	\item Lorem ipsum dolor sit amet
\end{itemize}


% Vår roll i projektet
\section{Roller}
\sloppy
\noindent Lorem ipsum dolor sit amet, consectetur adipisicing elit, sed do eiusmod tempor incididunt ut labore et dolore magna aliqua. Ut enim ad minim veniam, quis nostrud exercitation ullamco laboris nisi ut aliquip ex ea commodo consequat. Duis aute irure dolor in reprehenderit in voluptate velit esse cillum dolore eu fugiat nulla pariatur. Excepteur sint occaecat cupidatat non proident, sunt in culpa qui officia deserunt mollit anim id est laborum.

\subsection{Kundteam: I}
\noindent
Johan Barkfors, zba10jan@student.lu.se
\\Olof Spångö, ain09osp@student.lu.se
\\Daniel Jigin, elt11dji@student.lu.se
\\Max Andersson, elt11ma1@student.lu.se
\\Jacob Hedqvist, elt11jhe@student.lu.se
\\Andreas Wiberg, elt11awi@student.lu.se
\\Mattias Mellhorn, jcd11mto@student.lu.se


\subsection{Utvecklingsteam: F}
\noindent
Alexander Badju, adi10aba@student.lu.se (P3RM)	
\\*Fredrik Helander, gda10fhe@student.lu.se (EPM)
\\*Jonathan Klingberg, adi10jkl@student.lu.se (SCCVM/TDEVM)
\\*Jonathan Knorn, ada09jkn@student.lu.se (TDEVM)
\\*David Lundberg, adi10dlu@student.lu.se (QRM)
\\*Niklas Sjöberg, adi10nsj@student.lu.se (DRM)

\section{Potentiella intressenter}
\sloppy
\noindent

\subsection{Inre intressenter}
\noindent
\begin{itemize}
	\setlength\itemsep{0.1em}
	\item Lorem ipsum dolor sit amet
	\item Lorem ipsum dolor sit amet
	\item Lorem ipsum dolor sit amet
	\item Lorem ipsum dolor sit amet
	\item Lorem ipsum dolor sit amet
	\item Lorem ipsum dolor sit amet
	\item Lorem ipsum dolor sit amet
	\item Lorem ipsum dolor sit amet
	\item Lorem ipsum dolor sit amet
\end{itemize}
\subsection{Yttre intressenter}
\noindent
\begin{itemize}
	\setlength\itemsep{0.1em}
	\item Lorem ipsum dolor sit amet
	\item Lorem ipsum dolor sit amet
	\item Lorem ipsum dolor sit amet
	\item Lorem ipsum dolor sit amet
	\item Lorem ipsum dolor sit amet
	\item Lorem ipsum dolor sit amet
	\item Lorem ipsum dolor sit amet
	\item Lorem ipsum dolor sit amet
	\item Lorem ipsum dolor sit amet
\end{itemize}

\section{Aktivitetsplanering och leverabler}
\sloppy
\noindent
Projektet börjar med att utvecklingsgruppen skapar en v2 av kundgruppens Project Mission. Under detta arbete ska en dialog föras mellan de två grupperna så att man gemensamt kommer överens om produktens funktionalitet samt reder ut eventuella missförstånd.  
Därefter kommer projektet vara indelat i tre iterationer med tillhörande releaser, betecknade R1, R2 och R3. Var och en av releaserna ska delas in i två separata delar: System Requirements, som innehåller alla krav med tillhörande specifikationstekniker, och Project Experiences, som beskriver vår kravhanteringsprocess. Till iteration 2 tillkommer två leverabler, Validation Report och Validation Checklist. I Validation Report antar vår grupp kundrollen och validerar utvecklingsgruppens R2. Validation checklist innefattar att vår grupp tar utvecklarrollen och tar fram en kravvalideringschecklista åt kunden. I samband med R2 ges en presentation av vår arbetsgång och resultat. Se tabell 1 nedan för deadlines av leverabler.  

\begin{table}[htbp] % NICK: htbp är det nått som ska ändras? (kåppy pejstat)
  \begin{center}
    \begin{tabular}{l|ll}
      & \emph{Leverabel} & \emph{Deadline} \\ \hline
      Vecka 3      & Project Mission v2        & Måndag 17/11, 09.00 \\
      Vecka 4      & Release R1                & Måndag 24/11, 09.00 \\
      Vecka 6      & Release R2                & Måndag 8/12,  09.00 \\
                    & Validation checklist      & Måndag 8/12, 09.00 \\
                    & Validation report         & Fredag 12/12, 09.00 \\
      Vecka 7      & Conference presentation   & Söndag 14/12, 15.00 \\
                    & Release R3                & Söndag 21/12, 23.59 \\
     \hline               
    \end{tabular}
  \end{center}
  \caption{Lista över leverabler och dess deadlines}
\end{table}


\begin{table}[htbp]
    \begin{center}
        \begin{tabular}{ |c|c|c|c|c|c|c||c|c| }
        \hline
         & \emph{Week 2} & \emph{Week 3} & \emph{Week 4} & \emph{Week 5} &              \emph{Week 6} & \emph{Week 7} & \\
        \hline
        Alexander & 2 h & 16 h & 15 h & 17 h & 16 h & 16 h & 83 h\\
        David & 8 h & 13 h & 12 h & 17 h & 16 h & 17 h & 83 h\\
        Fredrik & 8 h & 13 h & 12 h & 17 h & 16 h & 17 h & 83 h\\
        Jonathan Kl & 6 h & 13 h & 14 h & 17 h & 16 h & 17 h & 83 h\\
        Jonathan Kn & 7 h & 13 h & 13 h & 17 h & 16 h & 17 h & 83 h\\
        Niklas & 9 h & 12 h & 12 h & 17 h & 16 h & 17 h & 83 h\\
        \hline
        \hline
         & 40 h & 80 h & 78 h & 102 h & 96 h & 102 h & 498 h\\
        \hline
        \end{tabular}
    \end{center}
    \caption{Lista över arbetsfördelning}
\end{table}


\end{document}

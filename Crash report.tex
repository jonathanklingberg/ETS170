%%%%%%%%%%%%%%%%%%%%%%%%%%%%%%%%%%%%%%%%%%%%%%%%%%%%%%%%%%%%%%%%%%%%%%
% How to use shareLaTeX: 
%
% You edit the source code here on the left, and the preview on the
% right shows you the result within a few seconds.
%
% Bookmark this page and share the URL with your co-authors. They can
% edit at the same time!
%
% You can upload figures, bibliographies, custom classes and
% styles using the files menu.
%
% If you're new to LaTeX, the wikibook is a great place to start:
% http://en.wikibooks.org/wiki/LaTeX
%
%%%%%%%%%%%%%%%%%%%%%%%%%%%%%%%%%%%%%%%%%%%%%%%%%%%%%%%%%%%%%%%%%%%%%%
%
% Template for PLoS
% Version 1.0 January 2009
%
% Edit the title below to update the display in My Documents
%\title{PLoS Journal Article}

\documentclass[10pt]{article}

% amsmath package, useful for mathematical formulas
\usepackage{amsmath}
% amssymb package, useful for mathematical symbols
\usepackage{amssymb}

% graphicx package, useful for including eps and pdf graphics
% include graphics with the command \includegraphics
\usepackage{graphicx}

% cite package, to clean up citations in the main text. Do not remove.
\usepackage{cite}

\usepackage{color} 

\usepackage[utf8]{inputenc}

\usepackage[none]{hyphenat}


\usepackage{titlesec}
\titlespacing\section{0pt}{12pt plus 4pt minus 2pt}{0pt plus 2pt minus 2pt}


% Use doublespacing - comment out for single spacing
\usepackage{setspace} 

% Text layout
\topmargin 0.0cm
\oddsidemargin 0.5cm
\evensidemargin 0.5cm
\textwidth 16cm 
\textheight 21cm

% Bold the 'Figure #' in the caption and separate it with a period
% Captions will be left justified
\usepackage[labelfont=bf,labelsep=period,justification=raggedright]{caption}
\captionsetup[table]{name=Tabell}
% Use the PLoS provided bibtex style
\bibliographystyle{plos2009}

% Remove brackets from numbering in List of References
\makeatletter
\renewcommand{\@biblabel}[1]{\quad#1.}
\makeatother

% Made sure "contents" is in swedish
\renewcommand{\contentsname}{Innehållsförteckning}

% Remove page numbers
%\pagenumbering{gobble}


% Leave date blank
\date{}

\pagestyle{myheadings}
%% ** EDIT HERE **


%% ** EDIT HERE **
%% PLEASE INCLUDE ALL MACROS BELOW

%% END MACROS SECTION

\begin{document}

% Title must be 150 characters or less

\begin{center}
{\Large
\textbf{CRASH - Comfort, Reliability and Self Handling}
}
\end{center}


\tableofcontents
\newpage

% Projektets bakgrund
\section{Bakgrund}
\sloppy
\noindent Utvecklingen av ett transportnät med en kontinuerligt förbättrad säkerhet och effektivitet är en för samhället mycket viktig utmaning och berör ekonomiska, hälsomässiga och, säkerhetsmässiga frågor. En allt större befolkning påverkar belastningen av vägnätet och förutsätter att det får en ökad kapacitet eller alternativt att transportflödena blir effektivare.

% Projektets mål   
\section{Mål}
\sloppy
\noindent Målet är att skapa ett styrsystem till en självkörande bil, som på ett säkert sätt kan köra på allmänna vägar. Bilen ska ta hänsyn till olika händelser i omgivning för att på ett säkert och för passageraren komfortabelt sätt anpassa sig till rådande trafiksituation. 

% Projektets funktionalitet
\section{Funktionalitet}
\sloppy
\noindent Systemet ska
\begin{itemize}
	\setlength\itemsep{0.1em}
	\item Alltid prioritera passagerarnas och omgivningens säkerhet högst.
	\item När en bilolycka är oundviklig ska styrsystemet agera på ett sånt sätt att skada minimeras och räddande av människoliv prioriteras.
	\item Följa svenska trafikregler.
	\item Ta hänsyn till händelser i omgivningen och rådande trafiksitution när aktuell rutt bestäms.
	\item Framföra fordonet på ett för passageraren komfortabelt sätt.
	\item Hantera bränsle sparsamt.
	\item Ta hänsyn till rådande och prognoser om framtida väderomständigheter vid säkerhetsvärdering.
	\item Ta hänsyn till bilens skick utifrån sensordata vid säkerhetsvärdering.
	\item Ta hänsyn till återstående räckvidd samt kunna lägga om rutten för att tanka vid behov.
	\item Vid konflikt prioritera i följande ordning; säkerhet, trafikregler, komfort och bränsleeffiktivitet.
	\item Autentisera förare med hjälp av en fingeravtryckssensor.
	\item Stödja inmatning av destination genom röststyrning eller en tillgänglig pekskärm på bilens instrumentpanel och ge feedback på detta både med ljudmeddelanden och visuellt.
	\item Kunna framföras utan passagerare.
	\item Tillåta att bilen körs manuellt om detta läge valts.
\end{itemize}


% Vår roll i projektet
\section{Roller}
\sloppy
\noindent Lorem ipsum dolor sit amet, consectetur adipisicing elit, sed do eiusmod tempor incididunt ut labore et dolore magna aliqua. Ut enim ad minim veniam, quis nostrud exercitation ullamco laboris nisi ut aliquip ex ea commodo consequat. Duis aute irure dolor in reprehenderit in voluptate velit esse cillum dolore eu fugiat nulla pariatur. Excepteur sint occaecat cupidatat non proident, sunt in culpa qui officia deserunt mollit anim id est laborum.

\subsection{Kundteam: grupp I}
\noindent
Johan Barkfors, zba10jan@student.lu.se (SCCVM)
\\Olof Spångö, ain09osp@student.lu.se
\\Daniel Jigin, elt11dji@student.lu.se
\\Max Andersson, elt11ma1@student.lu.se
\\Jacob Hedqvist, elt11jhe@student.lu.se
\\Andreas Wiberg, elt11awi@student.lu.se (P3RM)
\\Mattias Mellhorn, jcd11mto@student.lu.se


\subsection{Utvecklingsteam: grupp F}
\noindent
Alexander Badju, adi10aba@student.lu.se (P3RM)	
\\*Fredrik Helander, gda10fhe@student.lu.se (EPM)
\\*Jonathan Klingberg, adi10jkl@student.lu.se (SCCVM/TDEVM)
\\*Jonathan Knorn, ada09jkn@student.lu.se (TDEVM)
\\*David Lundberg, adi10dlu@student.lu.se (QRM)
\\*Niklas Sjöberg, adi10nsj@student.lu.se (DRM)

\section{Potentiella intressenter}
\sloppy
\noindent
Det finns potentiellt ett stort antal intressenter i det här projektet. Vi har valt att fokusera på de enligt oss största och viktigaste inre, respektive yttre intressenterna.

\subsection{Inre intressenter}
\noindent
\begin{itemize}
	\setlength\itemsep{0.1em}
	\item LADA - Nyckelkunderna för projektet, ställer krav och begär funktioner.
	\item Steve’s Angels - Utvecklarna av produkten
\end{itemize}
\subsection{Yttre intressenter}
\noindent
\begin{itemize}
	\setlength\itemsep{0.1em}
	\item Trafikverket - Ansvarar för den långsiktiga planeringen av infrastruktur i Sverige.
	\item Polismyndigheten - Polisen är ansvarig för att uppråtthålla trafikregler, och påverkas således av den här produkten.
	\item Riksdagen - Riksdagen beslutar om alla lagar i Sverige, och en lagändring skulle behövas för att tillåta förarlös teknik.
	\item Regeringen - Regeringen driver igenom de lagar som Riksdagen beslutar om, se ovan.
	\item Försäkringsbolag - Bilar utan förare, eller med endast passagerare (föraren ses som en passagerare när bilen körs automatiskt) medför nya problem kring försäkringar och ansvar i trafiken.
	\item Bilförare - Slutkonsumenter, gruppen som kommer köpa produkten.
	\item Transportstyrelsen - Utformar regler för bland annat biltrafik, kontrollerar hur de efterföljs och utfärdar körkort.
	\item Lantmäteriet - Utvecklar kartor över Sverige, vilket behövs för att vår produkt ska fungera.
	\item Trafikanalys - Förser beslutsfattarna med underlag för att ta beslut kring transportpolitik.
\end{itemize}

\section{Aktivitetsplanering och leverabler}
\sloppy
\noindent
Projektet börjar med att utvecklingsgruppen skapar en v2 av kundgruppens Project Mission. Under detta arbete ska en dialog föras mellan de två grupperna så att man gemensamt kommer överens om produktens funktionalitet samt reder ut eventuella missförstånd.  
Därefter kommer projektet vara indelat i tre iterationer med tillhörande releaser, betecknade R1, R2 och R3. Var och en av releaserna ska delas in i två separata delar: System Requirements, som innehåller alla krav med tillhörande specifikationstekniker, och Project Experiences, som beskriver vår kravhanteringsprocess. Till iteration 2 tillkommer två leverabler, Validation Report och Validation Checklist. I Validation Report antar vår grupp kundrollen och validerar utvecklingsgruppens R2. Validation checklist innefattar att vår grupp tar utvecklarrollen och tar fram en kravvalideringschecklista åt kunden. I samband med R2 ges en presentation av vår arbetsgång och resultat. Se tabell 1 nedan för deadlines av leverabler.  
\\
\\

\begin{table}[htbp] % NICK: htbp är det nått som ska ändras? (kåppy pejstat)
  \begin{center}
    \begin{tabular}{l|ll}
      & \emph{Leverabel} & \emph{Deadline} \\ \hline
      Vecka 3      & Project Mission v2        & Måndag 17/11, 09.00 \\
      Vecka 4      & Release R1                & Måndag 24/11, 09.00 \\
      Vecka 6      & Release R2                & Måndag 8/12,  09.00 \\
                    & Validation checklist      & Måndag 8/12, 09.00 \\
                    & Validation report         & Fredag 12/12, 09.00 \\
      Vecka 7      & Conference presentation   & Söndag 14/12, 15.00 \\
                    & Release R3                & Söndag 21/12, 23.59 \\

     \hline 
    \end{tabular}
  \end{center}
  \caption{Lista över leverabler och deras deadlines}
\end{table}

\begin{table}[htbp]
    \begin{center}
        \begin{tabular}{ |c|c|c|c|c|c|c||c|c| }
        \hline
         & \emph{Vecka 2} & \emph{Vecka 3} & \emph{Vecka 4} & \emph{Vecka 5} &              \emph{Vecka 6} & \emph{Vecka 7} & \\
        \hline
        Alexander Badju & 2 h & 16 h & 15 h & 17 h & 16 h & 16 h & 83 h\\
        David Lundberg & 8 h & 13 h & 12 h & 17 h & 16 h & 17 h & 83 h\\
        Fredrik Helander & 8 h & 13 h & 12 h & 17 h & 16 h & 17 h & 83 h\\
        Jonathan Klingberg & 6 h & 13 h & 14 h & 17 h & 16 h & 17 h & 83 h\\
        Jonathan Knorn & 7 h & 13 h & 13 h & 17 h & 16 h & 17 h & 83 h\\
        Niklas Sjöberg & 9 h & 12 h & 12 h & 17 h & 16 h & 17 h & 83 h\\
        \hline
        \hline
         & 40 h & 80 h & 78 h & 102 h & 96 h & 102 h & 498 h\\
        \hline
        \end{tabular}
    \end{center}
    \caption{Lista över arbetsfördelning}
\end{table}


\end{document}

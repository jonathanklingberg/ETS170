%%%%%%%%%%%%%%%%%%%%%%%%%%%%%%%%%%%%%%%%%%%%%%%%%%%%%%%%%%%%%%%%%%%%%%
% How to use shareLaTeX: 
%
% You edit the source code here on the left, and the preview on the
% right shows you the result within a few seconds.
%
% Bookmark this page and share the URL with your co-authors. They can
% edit at the same time!
%
% You can upload figures, bibliographies, custom classes and
% styles using the files menu.
%
% If you're new to LaTeX, the wikibook is a great place to start:
% http://en.wikibooks.org/wiki/LaTeX
%
%%%%%%%%%%%%%%%%%%%%%%%%%%%%%%%%%%%%%%%%%%%%%%%%%%%%%%%%%%%%%%%%%%%%%%
%
% Template for PLoS
% Version 1.0 January 2009
%
% Edit the title below to update the display in My Documents
%\title{PLoS Journal Article}

\documentclass[10pt]{article}

% amsmath package, useful for mathematical formulas
\usepackage{amsmath}
% amssymb package, useful for mathematical symbols
\usepackage{amssymb}

% graphicx package, useful for including eps and pdf graphics
% include graphics with the command \includegraphics
\usepackage{graphicx}

% cite package, to clean up citations in the main text. Do not remove.
\usepackage{cite}

\usepackage{color} 

\usepackage[utf8]{inputenc}

\usepackage[none]{hyphenat}


\usepackage{titlesec}
\titlespacing\section{0pt}{12pt plus 4pt minus 2pt}{0pt plus 2pt minus 2pt}


% Use doublespacing - comment out for single spacing
\usepackage{setspace} 

% Text layout
\topmargin 0.0cm
\oddsidemargin 0.5cm
\evensidemargin 0.5cm
\textwidth 18cm 
\textheight 25cm

% Bold the 'Figure #' in the caption and separate it with a period
% Captions will be left justified
\usepackage[labelfont=bf,labelsep=period,justification=raggedright]{caption}

% Use the PLoS provided bibtex style
\bibliographystyle{plos2009}

% Remove brackets from numbering in List of References
\makeatletter
\renewcommand{\@biblabel}[1]{\quad#1.}
\makeatother


% Remove page numbers
\pagenumbering{gobble}



% Change margins
\usepackage[top=2.5cm, bottom=2.5cm, left=2.5cm, right=2.5cm]{geometry}

% Leave date blank
\date{}

\pagestyle{myheadings}
%% ** EDIT HERE **


%% ** EDIT HERE **
%% PLEASE INCLUDE ALL MACROS BELOW

%% END MACROS SECTION

\begin{document}

% Title must be 150 characters or less

\begin{center}
{\Large
\textbf{Compromizr}
}
\end{center}




% Projektets bakgrund
\section*{Bakgrund}
\sloppy
\noindent I grupp är det ofta svårt att på ett demokratiskt sätt ta kollektiva beslut, till exempel om ett kompisgäng ska äta lunch på stan kan det ofta ta lång tid att enas om var man ska äta. Lösningen på problemet är Compromizr, en applikation som låter folk rösta och på ett effektivt sätt leder till att ett demokratiskt beslut tas. 

% Projektets mål   
\section*{Mål}
\sloppy
\noindent Vårt mål är en serverfri app-lösning där användarna ansluter till varandra ad-hoc, lägger in sina olika förslag och sedan röstar på varandras förslag för att nå ett gemensamt beslut.

% Projektets funktionalitet
\section*{Funktionalitet}
\sloppy
\noindent Applikationen ska ha följande funktionalitet: 
\begin{itemize}
	\setlength\itemsep{0.1em}
	\item Applikationen ska vara iOS-baserad
	\item Ingen registrering ska krävas för att använda appen
	\item Möjlighet att skapa både öppna och stängda frågegrupper där gruppens skapare accepterar eller nekar användare när de försöker gå med i gruppen.
	\item Anslutningen mellan användare ska ske via Bluetooth
    \item När applikationen startas ska användaren mötas av alternativen att antingen skapa en ny eller gå med i en befintlig grupp.
    \item Swipe-lösning för att rösta (tänk Tinder) där en svepning till vänster betyder att man röstar mot förslaget och svepning till höger betyder att man röstar för förslaget.
    \item Svaren skall vara anonyma och inte visas för andra användare
    \item Som resultat visas endast alternativet med flest röster. Skulle flera alternativ dela på förstaplatsen startas en ny omröstning med endast dessa alternativ. 
    \item Applikationen skall vara utbyggbar för att i framtiden kunna utvidgas med stöd för en klient/server-lösning så att användarna inte behöver vara på samma plats.
    \item Alla deltagare ska kunna skapa egna alternativ som sedan ingår i omröstningen.
\end{itemize}


% Vår roll i projektet
\section*{Roller}
\sloppy
\noindent Steve's Angels är Lunds marknadsledande applikationsutvecklande kollektiv med projekt så som den kritikerrosade navigeringslösningen Navibration. Vi har beslutat oss för att börja outsourca utvecklandet och ägna vår tid åt vår verkliga styrka - skapa revolutionerande produkter och erbjuda branschledande marknadsföring med hjälp av sylvassa reklamfilmer och keynote-presentationer.

Vi kommer agera produktägare i detta projekt där ni har som uppdrag att utveckla vår applikation. Som huvudsaklig leverabel förväntar vi oss en kravspecifikation utifrån de funktionella krav vi listat ovan. Vi äger alla immateriella rättigheter till alla dokument som skapas utifrån detta dokument.

% Deltagare samt dess email
\section*{Deltagare}
\noindent Alexander Badju, adi10aba@student.lu.se (Projektledare)	
\\*Fredrik Helander, gda10fhe@student.lu.se 
\\*Jonathan Klingberg, adi10jkl@student.lu.se (Kundkoordinator)
\\*Jonathan Knorn, ada09jkn@student.lu.se 
\\*David Lundberg, adi10dlu@student.lu.se 
\\*Niklas Sjöberg, adi10nsj@student.lu.se



\end{document}

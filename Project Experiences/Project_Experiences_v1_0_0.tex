%%%%%%%%%%%%%%%%%%%%%%%%%%%%%%%%%%%%%%%%%%%%%%%%%%%%%%%%%%%%%%%%%%%%%%
%
% If you're new to LaTeX, the wikibook is a great place to start:
% http://en.wikibooks.org/wiki/LaTeX
%
%%%%%%%%%%%%%%%%%%%%%%%%%%%%%%%%%%%%%%%%%%%%%%%%%%%%%%%%%%%%%%%%%%%%%%
%
% Template for PLoS
% Version 1.0 January 2009
%
% Edit the title below to update the display in My Documents
%\title{PLoS Journal Article}

\documentclass[10pt]{article}

% amsmath package, useful for mathematical formulas
\usepackage{amsmath}
% amssymb package, useful for mathematical symbols
\usepackage{amssymb}

% graphicx package, useful for including eps and pdf graphics
% include graphics with the command \includegraphics
\usepackage{graphicx}

% cite package, to clean up citations in the main text. Do not remove.
\usepackage{cite}

\usepackage{color} 

% inputenc package, allows the user to input accented characters directly from the keyboard
\usepackage[utf8]{inputenc}

\usepackage[parfill]{parskip}

% hyphenat package, can be used to disable all hyphenation in a document or in selected text within % the document
\usepackage[none]{hyphenat}

% tocbibind package, automatically adds the bibliography and/or the index and/or the contents
% etc.., to the Table of content listing. 
\usepackage[nottoc,notlot,notlof]{tocbibind}

%\usepackage{titlesec}
%\titlespacing\section{0pt}{12pt plus 4pt minus 2pt}{0pt plus 2pt minus 2pt}
\setcounter{secnumdepth}{5}
\setcounter{tocdepth}{5}

% Use doublespacing - comment out for single spacing
\usepackage{setspace} 

% Text layout
\topmargin 0.0cm
\oddsidemargin 0.5cm
\evensidemargin 0.5cm
\textwidth 16cm 
\textheight 21cm

% Bold the 'Figure #' in the caption and separate it with a period
% Captions will be left justified
\usepackage[labelfont=bf,labelsep=period,justification=raggedright]{caption}
\captionsetup[table]{name=Tabell}
% Use the PLoS provided bibtex style
\bibliographystyle{plos2009}

\newcommand\todo[1]{\textcolor{red}{#1}}
\begin{document}

% Remove brackets from numbering in List of References
\makeatletter
\renewcommand{\@biblabel}[1]{\quad#1.}
\makeatother


\pagestyle{myheadings}
%% ** EDIT HERE **

%% ** EDIT HERE **
%% PLEASE INCLUDE ALL MACROS BELOW

%% END MACROS SECTION



% Title must be 150 characters or less
\begin{titlepage}
\title{Project Experiences}
\author{Group F}
\date{\today}
\maketitle
\thispagestyle{empty}
\end{titlepage}

%\newpage
\tableofcontents
\thispagestyle{empty}
\newpage
\pagenumbering{arabic}

%%%%% Requirement engineering work %%%%%%%%% 
\section{Requirement engineering work}
\noindent The project will be split up into three different releases whose main objectives along with some project experiences are described below.

\todo{Style paranteser, ser lika dan ut som sub-subsections just nu. Eventuellt indent}

\subsection{Release 1}
The project started off with us setting up a group interview with the customer where we encouraged them to explain as much as possible about their ideas. After this meeting we realized the project scope would be too large to accomplish within the scope of this course. Furthermore we needed to figure out and focus on the most vital functionalities for this system.
To get a first overview of the system features we started by creating a mind map out of the material we had gathered from the initial customer meeting, with the use of this map we could cut off branches with features not vital for the system to reduce the project scope.
We had additional discussions with our customer about the eliminated branches, and in the end we managed to negotiate a bit more reasonable project scope.
\newline We've also created a context diagram to define the system interactions and different user roles.
The main objective for release 1 was to focus on elicitation techniques in order to get a grip of the system requirements, which are summarized in the System Requirements Specification or SRS document, see reference \cite{srs}.

\todo{Vi ifråasatte varför inte 4-hjuling, varför fingerprint osv.}

\subsection{Release 2}
\todo{Fyll ut åtminstone lite om validation prioritazion and so on}

%%%%% Elicitation %%%%%%%%% 
\section{Elicitation}
\sloppy
\noindent The process of finding and formulating requirements is called elicitation.
\todo{Fyll ut och skriv on "Is the process"}

\subsection{Elicitation techniques}
\todo{Fyll ut reflections men lämna diskussion till slutet}
Elicitation techniques are utilized when translating stakeholders' aspirations into requirements.
The elicitation techniques that we decided to use are described below.
Besides those techniques, we've also tried "observation" and "document study", but it didn't feel worth spending more time on those since they where time consuming and not that efficient at this point in the project.
Putting together a Focus group should be of great benefit but we don’t feel like we have the resources enough to accomplish this.
A domain workshop would be great but we don't feel like it's applicable since there exists no expert users for this domain as far as we know.
However for our next release we're going to try prototyping as a new elicitation technique.

\subsubsection{Stakeholder analysis}
In a stakeholder analysis different people which are needed to ensure the success of the project are first listed. Then these people are either interviewed or invited to a joint meeting in order to collect their ideas about the system. These ideas are then gathered into a sustainable model where as many of the stakeholders as possible gets satisfied with the result. The goal of the analysis is to find answers to the following questions:
\begin{itemize}
\item What goals do they see for the system?
\item Why would they like to contribute?
\item What risks and costs do they see?
\item What kind of solutions and suppliers do they see?
\end{itemize}

\paragraph{Motivation}
\hfill \break
It’s a simple way to summarize and to get an overview of all groups and peoples who probably will come in contact with this system. Since this list consists of different kinds of users, it's very important to listen to what each of them feels like important for them so that all parties gets satisfied with the final result.

Since this list consists of different kinds of users, it's important 
\paragraph{Method}
\hfill \break
We listed all of the people and organizations that are needed to ensure the success of the project. 
Inner stakeholders:
\todo{Översätt till engelska. DONE!}
\begin{itemize}
\item LADA - key customers, 
\item Steve's Angels - develops the product. 
\end{itemize}

Outer stakeholders:
\begin{itemize}
\item Trafikverket - Responsible for the long-term planning of infrastructure in Sweden.
\item Polismyndigheten - The police is responsible for enforcing traffic regulations, and thus influenced by this product.
\item Swedish parliament - The Parliament decides on all laws in Sweden, and a change in the law would be needed to allow driverless technology.
\item Swedish government - The government enforces the laws that Parliament decides on, see above.
\item Trafikanalys - Provides policy makers with a basis for making decisions about transport policy.
\item Transportstyrelsen - 
\item Insurance companies - Cars without drivers, or with only passengers (the driver is seen as a passenger when the car is running automatically) poses new problems concerning insurance and liability in traffic.
\item Car owners - end users
\item Lantmäteriet - Develops maps of Sweden, which is needed for our product to work.
\end{itemize}

Since most of the stakeholders are large organizations, we had to pretend what their interests in this system was and we came up with features of interest for each one of them.
\paragraph{Reflection}
\hfill \break
For example we found that one of the features that all of the stakeholders had in common was their major interests in enhanced traffic-security and further we realized the importance of implementing these features. We also realized that the insurance companies might lose income in case this system gets too secure and nobody buys traffic insurances anymore.

\subsubsection{Interviewing}
\todo{Korta ner stycket men oklart hur viktigt detta är. lämna till sist}
Interviewing is one of the most widely used eliciting techniques, but it has its pros and cons. It is a quite simple and straightforward technique that requires little planning and can be used in various situations. It is a good technique for getting information about the present work in the domain as well as present problems. It can also help identify where conflicts may lie, however, other techniques are needed to resolve the conflicts (for example, we used negotiation as a technique for resolving conflicts with our stakeholders/customer, this will be discussed later in the report). Interviewing is not as good at identifying the goals and critical issues.
When doing interviews it is important, or at least preferred, that you ask the questions to members from each user group. Management often choose to officially nominate representatives for a user group. However, experience have shown that representatives generally don’t know what is going on in the daily business. Therefore, it is recommended to interview other staff members as well. Interviews can be conducted with either individuals or in groups.
Generally there are two types of interviews: 

\begin{itemize}
    \item Structured - the interviewer has prepared a set of questions that needs to be answered. 
    \item Unstructured - the interviewer has not prepared any questions and instead           openly discusses what is expected from the system. \ldots 
\end{itemize}

\paragraph{Motivation}
\hfill \break

Since the Project Mission was quite vague and left out a lot of functionality, we had a lot of questions about the system that needed to be answered before we could start working on the requirements specification. We felt that we needed to get clarification fairly soon, and as interviewing requires little planning and work, and doesn’t take to long to perform, it was deemed a fitting technique to use. Interviews also allows the interviewer and interviewee(s) to discuss and explain the questions and answers, which can give even further clarification. It also gives us the possibility to ask follow-up questions to confirm our understanding.
\paragraph{Method}
\hfill \break
We felt that a structured interview where we prepared a list of questions to ask where most fitting when meeting with our customer. Unstructured interviews requires more experience and are harder to perform. However, we did leave room for follow-up questions to confirm our understanding of what the customer asked for and why they asked for it. 
\paragraph{Reflection}
\hfill \break
The interview elicitation technique has been of great success for our project. Even if we noticed some of the elicitation barriers it still feels like we've managed to gather the information needed to form a viable SRS and with little effort spent.

\subsubsection{Brainstorming}
Using brainstorming as an elicitation technique is an effective way of generating new ideas. During a brainstorm session it is of great importance not to criticize any idea. This creates an environment where no group member is afraid of being ridiculed and thus enable the group to produce new ideas at a higher rate.
The process of separating good ideas from bad ones is saved until after the brainstorming session.
\paragraph{Motivation}
\hfill \break
We decided to use this elicitation technique as it was a natural step in the beginning of the elicitation process. The technique enables us to quickly gather a set of ideas on how to handle new problems that occur.

\paragraph{Method}
\hfill \break
We held brainstorming sessions in a room where we had access to a whiteboard. During the sessions we wrote down the ideas and requirements we came up with on the board. The brainstorming sessions typically involved developers only.  
\todo{Lägg till under Method vilka stakeholders som var inblandande.DONE!}
\paragraph{Reflection}
\hfill \break
The technique allowed us not only to come up with new requirements but also enabled us, in the beginning of the elicitation process, to get an united overview of the system we are building.

\subsubsection{Negotiation}
The sole purpose of negotiation is to resolve conflicts. Conflicts can, for example, occur between supplier and customer, or between various stakeholders inside the organization. There are many ways of resolving conflicts, for example, having each party describe what they believe the other party wants and why they want it. However, when resolving a conflict in requirements engineering it is important to analyze the goals for each party. Conflicts are often about the solution, but the trick is to find solutions that don’t conflict and support everybody’s goals.
\paragraph{Motivation}
\hfill \break
During the interviews with our customer we had a few disagreements, or conflicts if you may, that needed to be resolved. We also experienced conflicts within our organization (project group) during elicitation and while choosing what tools to use in the project. Since negotiation is a good way of resolving conflicts, we chose to apply it to both cases.
\paragraph{Method}
\hfill \break
As the conflicts with our customer group occured during the interviews we addressed most of them directly by starting to negotiate. However, some of the conflicts needed further analysis of the goals for each party and therefore were addressed at a later meeting. The conflicts we had within our project group were addressed as they occured. For example, the issue with what tools to use in our project were solved by letting each project member describe why they wanted to use a certain tool and why it suited this project. 
\paragraph{Reflection}
\hfill \break
The negotiations resolved the conflicts we had internally in our project group as well as the ones we had with our customer group.

%%%%% ELICITATION BARRIERS %%%%%%%%%
\subsection{Elicitation barriers}
\todo{Byt namn på rubrik till Customer interactions}
\todo{Eliciteringsbarriärerna för strikta, gör om till flödande text och mindre teori mer egna erfarenheter}
\\
When converting stakeholders aspirations into requirements, we'll simply have to ask them questions about the purpose of the new system. During this process elicitation barriers often show up where the stakeholders ideas are hard for them to explain or easily misunderstood in any other way.
During the project elicitation phase we experienced some of the most common elicitation barriers, some examples are described below.

\subsubsection{Customer changes their minds}
One of the things that was made clear after our first interview with our customer was that the system should focus on a private audience. Just a few days later we had another discussion with them, and now they had changed their minds and decided it should be possible to integrate the system into a taxi-activity for enterprise purposes.

\subsubsection{Luxury requirements}
During our first interview we felt that the customer came up with too many requirements. Some of them were essential, while others were just “nice-to-have”, or so called luxury requirements. For example, they wanted the possibility for the autonomous cars to form a train so that the road network would be used more efficiently. While this was a nice feature to have, we did not feel that it was a necessity for achieving their main goals. This and similar disagreements were solved by further negotiations and reasoning. 

\subsection{Elicitation reflections}
\todo{Fyll på med diskussion men låg prio. lämna till sist}


%%%%% TOOL REFLECTION %%%%%%%%%
\section{Tool reflections}
\noindent
\todo{Lägg till reflektioner}


%%%%% INDIVIDUAL STATEMENTS %%%%%%%%%
\section{Individual statements}
\noindent
\subsection{Alexander Badju, P3RM}
As the project leader, Alexanders contribution to this weeks work has consisted of organizing some of the work load as well as having last draft of the reports. Besides the regular group work, such as creating the context diagram, meetings etc. He has mostly worked with developing and eliciting system requirements as well as developing scenarios. 
\todo{Se bettans kommentarer}
\subsection{Fredrik Helander, EPM}
Fredrik has participated in group meetings and discussions as well as been active in the requirement elicitation process, he has also formulated tasks and requirements.  
\subsection{Jonathan Klingberg, SCCVM/TDEVM}
Jonathan has written some parts of the project experiences document, besides this and the regular project work such as being active at meetings and group discussions, he has also been working with setting up tools including some initial tool guidance for the other project members.
\subsection{Jonathan Knorn, TDEVM}
Jonathan has mainly been working with requirements both elicitation, dependecies and mapping tools. He has also been active in the common group work and been active at meetings.
\subsection{David Lundberg, QRM}
David has been working mostly with the requirements, been active at the meetings, and worked with scenarios as well as finding elicitation methods.
\subsection{Niklas Sjöberg, DRM}
Niklas has taken part in group meetings as well as customer meetings. Attended the reqT introduction meeting. Worked with elicitation techniques and requirements elicitation, and written parts of the project experiences document. 

%\addcontentsline{toc}{section}{References}
\begin{thebibliography}{1}
\bibitem{pmv2} Crash Project Mission version 2. 
\bibitem{srs} Crash Software Requirements Specification version 1. 

\end{thebibliography}

\end{document}
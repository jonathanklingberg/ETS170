%%%%%%%%%%%%%%%%%%%%%%%%%%%%%%%%%%%%%%%%%%%%%%%%%%%%%%%%%%%%%%%%%%%%%%
%
% If you're new to LaTeX, the wikibook is a great place to start:
% http://en.wikibooks.org/wiki/LaTeX
%
%%%%%%%%%%%%%%%%%%%%%%%%%%%%%%%%%%%%%%%%%%%%%%%%%%%%%%%%%%%%%%%%%%%%%%
%
% Template for PLoS
% Version 1.0 January 2009
%
% Edit the title below to update the display in My Documents
%\title{PLoS Journal Article}

\documentclass[10pt]{article}

% amsmath package, useful for mathematical formulas
\usepackage{amsmath}
% amssymb package, useful for mathematical symbols
\usepackage{amssymb}

% graphicx package, useful for including eps and pdf graphics include graphics with the command \includegraphics
\usepackage{graphicx}

% Tells latex that the images are kept in a folder named figures under the current directory. 
\graphicspath{ {figures/} }

% cite package, to clean up citations in the main text. Do not remove.
\usepackage{cite}

\usepackage{color} 

% inputenc package, allows the user to input accented characters directly from the keyboard
\usepackage[utf8]{inputenc}

\usepackage[parfill]{parskip}

% hyphenat package, can be used to disable all hyphenation in a document or in selected text within the document
\usepackage[none]{hyphenat}

% The package hyperref provides LaTeX the ability to create hyperlinks within the document.
\usepackage{hyperref}

% tocbibind package, automatically adds the bibliography and/or the index and/or the contents etc.., to the Table of content listing. 
\usepackage[nottoc,notlot,notlof]{tocbibind}

\usepackage{titlesec}
\titlespacing{\paragraph}{15pt}{*4}{*1.5}

%Sets the length of indents when beginning on a new line in a paragraph. 
\setlength{\parindent}{1cm}

% The todonotes package allows you to insert to{do items in your docu-ment. At any point in the document a list of all the inserted to{do items can be listed with the \listoftodos command
\usepackage{todonotes}


\usepackage[toc,page]{appendix}

%\titlespacing\section{0pt}{12pt plus 4pt minus 2pt}{0pt plus 2pt minus 2pt}
\setcounter{secnumdepth}{5}
\setcounter{tocdepth}{5}

% Use doublespacing - comment out for single spacing
\usepackage{setspace} 

% Text layout
\topmargin 0.0cm
\oddsidemargin 0.5cm
\evensidemargin 0.5cm
\textwidth 16cm 
\textheight 21cm

% Bold the 'Figure #' in the caption and separate it with a period
% Captions will be left justified
\usepackage[labelfont=bf,labelsep=period,justification=raggedright]{caption}
\captionsetup[table]{name=Tabell}
% Use the PLoS provided bibtex style
\bibliographystyle{plos2009}

%\newcommand\todo[1]{\textcolor{red}{#1}}
\begin{document}

% Remove brackets from numbering in List of References
\makeatletter
\renewcommand{\@biblabel}[1]{\quad#1.}
\makeatother


\pagestyle{myheadings}
%% ** EDIT HERE **

%% ** EDIT HERE **
%% PLEASE INCLUDE ALL MACROS BELOW

%% END MACROS SECTION



% Title must be 150 characters or less
\begin{titlepage}
\title{Project Experiences}
\author{Group F}
\date{\today}
\maketitle
\thispagestyle{empty}
\end{titlepage}

%\newpage
\tableofcontents
\thispagestyle{empty}
\newpage
\pagenumbering{arabic}

%%%%% Requirement engineering work %%%%%%%%% 
\section{Requirement engineering work}

\subsection{Release 1}
The project started off with us setting up a group interview with the customer where we encouraged them to explain as much as possible about their ideas. After this meeting we realized the project scope would be too large to accomplish within the given time of this course. Furthermore we needed to figure out and focus on the most vital functionalities for this system.
To get a overview of the system features we started by creating a mind map out of the material we had gathered from the initial customer meeting. With the use of this map we could cut off branches of features not vital for the system to reduce the project scope.
We had additional discussions with our customer about the eliminated branches, and in the end we managed to negotiate a more reasonable extent of the project.
\newline We've also created a context diagram to define the system interactions and different user roles.
The main objective for release 1 was to focus on elicitation techniques in order to get a grip of the system requirements, which are summarized in the System Requirements Specification or SRS document, see reference \cite{srs}.

\subsection{Release 2}
In release two we put up a a walk-through together with some customer representatives. Here we simulated some scenarios described in the task section in our SRS. For the SRS we've mainly focused on improving the quality of existing requirements, create more requirements, scenarios as well as tasks. Moreover, we have also made an effort visualizing the different dependencies of requirements, which wasn't that intuitive in our previous release. We've also created a checklist form which our customer will fill in during their validation of our SRS and will help us to address the most critical issues.
For this release we've also created a data model and some improvements to our previous context diagram which can be found in the SRS \cite{srs}.

\subsection{Release 3}
For the third project release we've been focusing on the prioritization of the requirements.

%%%%% Elicitation %%%%%%%%% 
\section{Elicitation}
\sloppy
\noindent
\todo{Intro till elicitation}
In requirements engineering elicitation is an important process since its at this stage requirements and needs for the system becomes apparent. There are many different elicitation techniques and methods that can be used, the tricky part is knowing and recognizing which techniques are preferable in different situations and for certain purposes.

\subsection{Elicitation techniques}
The elicitation techniques we've used in the project are described in this section. Besides from the techniques used we also discussed putting together a focus group since we thought we would have benefit of it. The reason why we didn't use it was that we came to the conclusion that it would require too much work to organize and that it was better for us to use this time to get experience in more techniques instead. We also discussed having a domain workshop, but had problem defining expert users in this domain since autonomous cars aren't a reality yet.

\subsubsection{Stakeholder analysis}
We decided to make a stakeholder analysis since is a simple way to get an overview of all stakeholders and their opinions. Since the stakeholders have different relations to the system it was important to take every ones opinion into account to assure that all parties would be satisfied with the final result. To identify the possible stakeholders we had an internal brainstorming session within the group, after this we discussed the different possible stakeholders and agreed on a final list.

\paragraph{Method}
\hfill \break
Usually when conducting a stakeholder analysis you invite the different stakeholders to a joint meeting, alternatively interview them one by one. Since almost all of the stakeholders we identified were large organizations it was hard to gather them for a meeting and too time consuming to interview them. As the Internet is a great source of information, we instead chose to study the organizations and formed our own understanding of why they would have any interest in our product. 
\todo{Skriva kort förklaring till vad vi gjorde med icke-organisationerna (Användare etc)?}

\paragraph{Reflection}
\hfill \break
While conducting the stakeholder analysis we found that one of the features that most of the stakeholders had in common was their interest in enhanced traffic-security. We also realized that driving schools might lose income in case our system would get widely used and people might loose interest in getting drivers licenses. A list of all the stakeholders identified and their interests can be found in the SRS\cite{srs}. 

\noindent As mentioned earlier we found it very hard and time consuming to contact all stakeholders since most of them are large organizations. Instead we decided to try to figure out their interests in our system and came up with features of interest for each one of them by ourself. This means that our view of their interests in the product may not fully correspond with their actual views, but for the scope of this project we thought it was a necessary compromise.

\subsubsection{Interviewing}
Since the project mission was quite vague and left out a lot of functionality, we had a lot of questions about the system that needed to be answered before we could start working on the requirements specification. We felt that we needed to get clarification from our key customer fairly soon, and as interviewing requires little planning and doesn't take long to perform, it was deemed a fitting technique to use. Interviews also allows the interviewer and interviewees to discuss the questions, which can give even further clarification. It also gave us the possibility to ask follow-up questions to confirm our understanding.

\paragraph{Method}
\hfill \break
As unstructured interviews requires a lot of experience and are difficult to perform, we felt that a structured interview was more fitting when meeting with our key customer. We prepared a list of questions to ask, but left room for follow-up questions to confirm and deepen our understanding of what the customer where asking for and why they asked for it. During both our interviews, 2-3 persons from our customer group attended. To make sure that everyone in our group had a clear view of what the customer wanted, all project members attended these interviews. The questioning was performed mainly by two members, with a third member taking notes. The rest of the group listened and jumped in with questions if they felt something was unclear. To validate our notes from the first interview we sent our notes to the customer for approval. Before the second interview we sent our questions to the customer beforehand so they could review them and prepare before the meeting.

\paragraph{Reflection}
\hfill \break
We found the interview elicitation technique valuable for our project since we got a clearer view both of what the customers wanted and why they wanted it. For instance, before our first interview we did not know  if they wanted the system to work in the entire world or just Sweden or why the system should take reach in to consideration.These and other uncertainties were made clear during the interviews. \\
\indent Even if we noticed some of the elicitation barriers it still feels like we managed to gather the information needed for us to be able to begin forming a viable SRS. The negotiation barriers are discussed further in section 2.2. \todo{Fixa riktig referens} 

\subsubsection{Brainstorming}
We decided to use brainstorming as our initial elicitation technique as it was a natural step in the beginning of the elicitation process. The technique enabled us to quickly gather a set of ideas on how to handle new problems that might occur.

\paragraph{Method}
\hfill \break
Most of our brainstorming sessions were held in a room where we had access to a whiteboard which proved to be very helpful. During the sessions we wrote down the ideas and requirements we came up with on the board and soon got a great overview of the project scope. The brainstorming sessions typically involved developers only. However, during our first customer meetings we held discussions that had somewhat of a brainstorming nature together with the customer as well. Ideas from these sessions where summarized in a document. We then had internal discussions within the development team to decide which ideas we found realistic and which ideas will require further negotiation with the customer.
\paragraph{Reflection}
\hfill \break
The technique allowed us not only to come up with new requirements but also enabled us, in the beginning of the elicitation process, to get an united overview of the system we are building. Not to criticize each others ideas can be difficult, especially i a group of six old friends.

\subsubsection{Negotiation}
The sole purpose of negotiation is to resolve conflicts. Conflicts can, for example, occur between supplier and customer, or between various stakeholders inside the organization. There are many ways of resolving conflicts, for example, having each party describe what they believe the other party wants and why they want it. However, when resolving a conflict in requirements engineering it is important to analyze the goals for each party. Conflicts are often about the solution, but the trick is to find solutions that don’t conflict and support everybody’s goals.
\paragraph{Motivation}
\hfill \break
During the interviews with our customer we had a few disagreements, or conflicts if you may, that needed to be resolved. We also experienced conflicts within our organization (project group) during elicitation and while choosing what tools to use in the project. Since negotiation is a good way of resolving conflicts, we chose to apply it to both cases.
\paragraph{Method}
\hfill \break
As the conflicts with our customer group occured during the interviews we addressed most of them directly by starting to negotiate. However, some of the conflicts needed further analysis of the goals for each party and therefore were addressed at a later meeting. The conflicts we had within our project group were addressed as they occured. For example, the issue with what tools to use in our project were solved by letting each project member describe why they wanted to use a certain tool and why it suited this project. 

\paragraph{Reflection}
\hfill \break
The negotiations resolved the conflicts we had internally in our project group as well as the ones we had with our customer. Negotiations where important in the process of narrowing the scope of the customers requested functionality. We deemed some functionality as unnecessary and, in some occasions, unrealistic.


%%%%% ELICITATION BARRIERS AND CUSTOMER INTERACTION%%%%%%%%%
\subsection{Customer Interaction}
When converting stakeholders aspirations into requirements, we'll simply have to ask them questions about the purpose of the new system. During this process elicitation barriers often show up where the stakeholders ideas are hard for them to explain or easily misunderstood in any other way.
During the project elicitation phase we experienced some of the most common elicitation barriers, some examples are described below.

\subsubsection{Customer changes their minds}
One of the things that was made clear after our first interview with our customer was that the system should focus on a private audience. Just a few days later we had another discussion with them, and now they had changed their minds and decided it should be possible to integrate the system into a taxi-activity for enterprise purposes. The customer is always right but since LADA only is our key customer we always have the option to deny their requests. We want our key customer to be satisfied with our solution and therefore it's important to take their request serious before consider denying it. There is no way to prevent the customer from changing their mind and this is one of the things that makes the life of a software requirement engineer so varying since you never know what crazy ideas the customer might come up with. However if the SRS is baselined we'll tell the customer to make a change request and then we'll do an impact analysis before accepting/denying the requested change.

\subsubsection{Luxury requirements}
During our first interview we felt that the customer came up with too many requirements. Some of them were essential, while others were just “nice-to-have”, or so called luxury requirements. For example, they wanted the possibility for the autonomous cars to form a train so that the road network would be used more efficiently. While this was a nice feature to have, we did not feel that it was a necessity for achieving their main goals. This and similar disagreements were solved by further negotiations and reasoning. 

\subsection{Elicitation reflections}
During the elicitation process we asked our customer the question "why?" multiple times. Our customer was very prone to tell us how they wanted us to fix their problems, this is a natural behaviour for customers but it's up to us to question their solutions and focus on their problem rather than their proposed solutions. If we manage to do this we're probably ending up with a system-solution that's more suitable for their needs. But off course, if we cannot convince the customer our solution is better than theirs, then we might have to give up our solution.
For example in our case the customer in the beginning told us they wanted some sort of parental control to be activated on the system but after further investigation we found that what they really wanted was just another user role for passengers who's not allowed to change destination.
\newline
\indent
\todo{Kan det här vara något som passar in på bettans kommentar om uppfyllnad av betygskritrie 4E?}
In the process of eliciting requirements we discovered new problems and issues along the way when exploring different possibilities with the CRASH-system. Since the scope for the project is so large, it felt like we were constantly finding new fields to discover, this lead us to look into and use different elicitation techniques to find suitable requirements for different situations. We mostly did this through reflections, brainstorming, discussions with each other as well as complementing the process with writing down scenarios. E.g. When determining the requirement that states that the car may break traffic laws if an accident is imminent, we used a scenario and brainstormed how such a scenario might play out. 

%%%%% REQUIREMENTS SPECIFICATION SECTION %%%%%%%%%
\section{Specification}
This section describes our work on the requirements specification as well as reflections on why we used certain methods. 


\subsection{R1}
\todo{Jag vet att engelskan här är ganska dålig, kör en omskrivning senare. Vill bara få ner all text först}
After we had our first interview with our customer, as well as done some research on existing autonomous cars, we had an improved understanding of what the customer wanted. There were of course still a few questions that were left unanswered, however, we felt that we had enough information to begin documenting some requirements. Therefore, the whole group sat down together and conducted a brainstorming session were we elicited some more requirements. The brainstorming session gave us a long list of requirements, but they were not divided into different categories requirements. So, to divide the requirements into different types we used a table structure where we had three main requirement levels; Functional, Data and Quality. We choose goal, domain and product as requirement sub-levels. See figure \ref{fi:classification} in the appendix to get a better view of our thought process. We are fully aware that the text is very small in the picture, but the requirement categories can still be seen. We got the following requirement types:  
\begin{itemize}
\item Functional goal requirements
\item Functional domain requirements
\item Functional product requirements
\item Data domain requirements
\item Data product requirements
\item Quality goal requirements
\item Quality domain requirements
\item Quality product requirements
\end{itemize}
Our experience was that it was hard to divide the requirements into categories, and we felt quite uncertain of where to put the requirements throughout the process. It was especially hard to determine if the requirements should be on product level or domain level. We were also uncertain whether this was a good way or not of dividing the requirements into different types. But we choose to keep it this way for R1 to discuss it further with our supervisor. 
After we divided our requirements into different categories we gave them the following identification numbers: XYyN. Where X is the first character of the requirement level, i.e., Functional, Data or Quality. Yy is the first and second character of the sub requirement level, i.e., Goal, Domain or Product. N is the requirement number, which is incremented by one for each requirement in each category. We chose to use this indexing method since some of the group members had used it in the PUSS course, and felt that it was a good and simple method to use. The drawback with this method is that the requirement name does not tell anything about what the requirement is about. Instead, we could have used strings as name for the requirements, but we felt that it would have been too time consuming to fit in our schedule for the first release. 
\newline
\indent For the first release we decided to document all of our requirements as features. Feature requirements are quite straightforward, and easy to formulate and change, and as many of the requirements we had produced so far were not fully specified, we felt that the feature style was our best option for now. At this time we prioritized to develop requirements that covered large parts of the system rather than working with different requirements styles.
\newline
\indent Even if it can be quite time consuming to develop a context diagram we choose to allocate a lot time on it in the beginning of the project. This because it can be a great way to show which interfaces the product should communicate with, as well as great way for the customer to see if any interface is missing. Context diagram can also show product scope, and as we believed that the scope described in PMv1 \cite{pmv1} was too wide to fit this course, this was a good way to make sure that our customers agreed with our reduced project scope. During our work on the context diagram we experienced a few difficulties. In particular, it was hard to identify which interfaces that should be included in the context diagram and if the content we added was on product or domain level. 
\newline
\indent
\indent We decided early on that we should give reqT a chance, so all our requirements were moved into a reqT model. We had earlier in the specification process noted all dependencies between requirements, and with the help of reqT we tried to visualize this. We ended up with a list of requirements and a huge graph that was hard to read. As the list only consisted of the requirement names, and we had chosen to name our requirements with numbers, it had a bad overview. 

\subsection{R2}
All of our requirements from our specification process for R1 were specified as feature requirements, and frankly, many of them were poorly specified. Therefore, we started our work on R2 with reviewing all of our requirements in the requirements specification. As expected, there was room for a lot of improvement and the review showed that some of the requirements, due to their complexity, needed more information and description. As a result, we introduced four new styles in our requirements specification; task descriptions, scenarios, virtual windows and E/R-diagrams. At one point while specifying requirements, we found ourselves not knowing how to handle certain user situations with various outcomes. Therefore we decided to document some scenarios and tasks to help ourselves clarify the situation, as well as make it easier for the developers to understand.
\newline
\indent As we suspected while working on the first release, we had made an excessive categorization of the requirements that was quite hard to understand for people that had not been part of the development of the requirements specification. It also made After feedback from the supervisor and discussion within the group, we instead chose to use the following categories: 
\begin{itemize}
\item Goal requirements
\item Domain requirements
\item Quality requirements
\item Product requirements
\begin{itemize}
\item Functional product requirements
\item Data product requirements
\end{itemize} 
\item Design requirements
\end{itemize}
We were still not certain that this was a good categorization as it does not give a good overview over the system and can be quite hard to follow. So for the upcoming release we planned to divide the requirements into new categories that give more information of what is included in the system. This also gave us a chance to review our requirements once more. But for R2 we chose to allocate more time on other areas of the specification. 
\newline
\indent For R2 we continued to work on the context diagram, as it needed a few changes.  Some of the events/tasks needed to be rewritten so they were expressed on domain level. This was actually a problem all through the SRS as we often focused on a product solution instead of working on the domain level. 

\subsection{R3}
We received our supervisor Elizabeth's comments and thoughts of our R2 by e-mail before having the meeting regarding the R2 with her in person. We took all the comments we received and put them on a Kanban board so we had a good and clear overview.

After the meeting we continued to work with R3. We began with reorganizing our requirements based on functions. All requirements that belonged to a certain function where grouped together. After the reorganization was done we split up into two persons/group and each group was responsible for a function and all requirements corresponding to that specific section. 
The sections we decided to categorize our requirements are as follows:
\begin{itemize}
\item Goals
\item General
\item Safety
\subitem Emergency break
\subitem Traffic accidents
\subitem Weather and traffic information
\subitem Permissions
\item Surroundings
\subitem Positioning
\subitem Traffic and maps
\subitem Sensors
\item Traffic rules
\item Users
\subitem Permissions
\subitem Storage of data
\item Manual driving
\subitem Activation
\subitem Personal safety
\subitem Road safety
\item I/O
\subitem Voice control
\subitem Touch screen
\item Remote
\item Updates
\item Data storage
\item Car status
\subitem Service log
\subitem Status and service
\item Authentication
\item Comfort
\end{itemize}



%%%%% TOOL REFLECTION section %%%%%%%
\section{Tool reflections}
\noindent
\subsection{ReqT}
ReqT is an open source tool which is designed for easy development of complex system requirements specifications during requirement engineering. ReqT provides many functionalities that would be close to impossible to achieve with an everyday word processing tools such as Word and Excel. ReqT also have many import and export functionalities that are beneficial when used with other tools such as LaTeX and Excel
\subsubsection{Benefits}
ReqT allows us to build many complex dependencies
\subsubsection{Problems}

\subsection{ShareLaTeX?}

\subsection{Slack?}
"So yeah, we tried Slack..."

%%%%% CHECKING AND VALIDATION %%%%%%%
\section{Checking and validation (CRASH)}
\noindent
During the development of a software requirement specification it's important to make sure the solution that is proposed in the specification are sustainable also in reality and that it lives up to the customers expectations. Therefore it's very important to perform some validation techniques to ensure the quality of the final SRS.
\subsection{Tests}
The best way to validate the requirements specification during its development is to perform tests. The customer or some experts can act as a user of the system who perform certain task in different approaches depending on the test method. These tests makes sure that the system-solutions proposed in the SRS also works in reality and that the developers of the SRS are on the same page as the customer.

\subsubsection{Walk-through}
In a walk-through some of the test scenarios are carried out with expert users and any problems or reflections are documented and used for further development of the SRS. If paper slips with current data are used in the system interaction then it would've been called a simulation.

\paragraph{Motivation}
\hfill \break
We decided to use this validation technique as it is an easy way for us to check so that we're building requirements for a system suited for our customers wishes. The technique enables us to easily make sure we're still on the same page about the system performances.

\paragraph{Method}
\hfill \break
A walk-through is best be carried out with some experts in the domain but since we don't know any experts (there doesn't really exist any) we simply had two of our customer representatives to carry out this walk-through with us.
We had prepared a number of scenarios from some of the tasks that are described in the SRS. These scenarios where then simply traversed through one by one and during the walk-through the customer representatives were allowed to give comments about our solutions and also to point out special cases we might not have thought of and which were important to them.
The walk-through focused on different authentication problematics such as picking up unauthorized children, lending the car to a friend overnight and tell the car to pick up a person at a specified time and place.

\paragraph{Reflection}
\hfill \break
During the walk-through most of our solutions were approved by the customer representatives and they even sounded very satisfied with them.
They also pointed out some special scenarios as for example when the car is running out of gas on it's way to a pick-up spot.
This technique allowed us not only to validate our existing tasks and requirement but also worked as another elicitation technique since we also came up with some new requirements.

\subsection{Validation Checklists}
To write a validation checklist that our customer can read while checking the requirements is a good way to ensure the quality of the SRS. Here the authors gets the opportunity to point out some of the things that's been especially complex during development, instead of hiding them under the carpet they are highlighted. This gives the customer a great overview of the complexity in their ordered system requirement specification and in turn can point out some of the things that needs further development before the SRS is signed.

\subsubsection{Create a validation checklist form to the customer}
To R2 we've created a validation checklist which will be used by our customer to validate our SRS. This list is tailor made to check some of the biggest challenges we've expirienced with our SRS. The customer goes through this list and reads our SRS simultaneously to check to which level the SRS fulfills each attribute in the list.
\todo{funderar på att ta bort dessa rubriker}

\subsubsection{Reflection}
There were unfortunately some problems regarding the structure of our validation checklist. Some of the content that we asked the customer to validate were left out of the SRS on purpose. In the checklist we requested validation of deliverables such as installation documents, documentation regarding support and maintenance together with functionality that was never planed to be delivered.
This did not prevent us from receiving the feedback we needed as the checklist covered more relevant content as well. However, handing a validation checklist with partly irrelevant content, that is not covered in the SRS, to the customer might give the impression that developers are not on the right track. 
\subsubsection{Feedback conclusions}
The feedback from our customer were of great help and the following list describes how their feedback led to improvements in our specification.
\begin{itemize}
\item "Some business goals have been defined but in vague terms (Maintain and expand market share). Business goals should be more concrete with metrics to make assessments of the results"

Some existing business goals were improved according to the feedback, some were completely removed but we also added some additional goal requrement for release 3.
\item "Trace requirement -> goal" 

Was fixed by adding more goals.
\item "Move goal requirements higher up in the document" 

This was just a mistake which was fixed for release 3.
\item "Is it possible that the car might enter "unsafe state" during driving?" 

Was solved by improving the definition of unsafe state.
\item "In general, very strict quality requirements. Might be rather expensive to implement". and "Almost all of the quality requirements seem unrealistic, e.g. instant touch screen response (Qu19)"

All quality requirements have been re-written in order to put them on a more reasonable level.
\end{itemize}

\noindent We received useful feedback on functional and quality requirements. The validation checklist also solved some uncertainties regarding business goals.

\section{Checking and validation (Compromizer)}

\subsection{Validation Report}
After release 2 we evaluated the SRS of Compromizer in two different approaches. First of all they had created a checklist where they highlighted some of the problems they wanted us to look into extra carefully. This list was of great help and also gave us the opportunity to leave additional comments on our observations during the validation process. However this list didn't feel enough and it felt complicated to describe some of our comments and we therefore decided to setup a validation meeting with our developers to make sure they agreed on our concerns.

\subsubsection{Validation Checklist}
The developers of Compromizr send us a checklist of observation tasks along with their SRS, see Appendix \cite{CompromizrChecklist}. They wanted us to evaluate their SRS and take extra care to the tasks listed in their checklist, we were also able to leave comments on those tasks needed further descriptions. We did this evaluation by splitting up their checklist in five pieces and then four of us took one part each and started evaluating the SRS with particular focus on the chosen items. The last part of the checklist was about the overall structure of the specification and this was filled by all of us.
When we had finished the checklist we still had some comment about missing and ambiguous requirements. We decided a validation meeting with the developers was be necessary.

\subsubsection{Validation meeting}
During the validation process we found more comments than could fit into their validation checklist so we decided to setup a meeting with them where we were able to go through some of the important notes we had taken in a more detailed level in order to make sure they understood our concerns. During this meeting we documented the notes both parties agreed on and the summary is found in Appendix \cite{CompromizrValidationMeeting}. We also ranked the issues for criticality A to C where A is severe and C just a note.


%%%%% PRIORITIZATION %%%%%%%%%
\section{Prioritization}

\subsection{Top-ten requirements}
\subsubsection{Motivation}
\subsubsection{Method}
\subsubsection{Reflection}

\subsection{The 100-dollar test}
\subsubsection{Motivation}
\subsubsection{Method}
\subsubsection{Reflection}

\todo{Add text of how we used the MindMap to prioritize in the beginning of the project.Could be good to mention that we sent it to the customer for validation (should probably be under validation)}

%%%%% INDIVIDUAL STATEMENTS %%%%%%%%%
\section{Individual statements}
\noindent
\subsection{Alexander Badju, P3RM}
Alexander's contribution has consisted of organizing some of the work load as well as being responsible for the last draft of the reports. Besides regular group work and taking part of meetings he has mostly worked with developing and eliciting all types of system requirements as well as developing and altering scenarios. 
\subsection{Fredrik Helander, EPM}
Fredrik has participated in group meetings and discussions as well as been active in the requirement elicitation process. He has also formulated tasks and requirements, been working on scenarios, as well as documenting project experiences.
\subsection{Jonathan Klingberg, SCCVM/TDEVM}
Jonathan has written parts of the project experiences document, besides this and the regular project work such as being active at meetings and group discussions, he has also been working with setting up tools including some initial tool guidance for the other project members.
Jonathan was responsible for setting up a validation technique with the customer, this ended up in a walk-through which result was of great benefit during internal validation of the second release SRS.
\subsection{Jonathan Knorn, TDEVM}
Jonathan has mainly been working with requirements both elicitation, dependencies and mapping tools. 
He has also been active in the common group work and been active at meetings.
\subsection{David Lundberg, QRM}
David has been working mostly with the requirements. He has also been active at the meetings, worked with scenarios as well as finding elicitation methods. He has worked with all requirements categories with a focus on goal and quality. 
\subsection{Niklas Sjöberg, DRM}
Niklas has taken part in group meetings as well as customer meetings. Attended the reqT introduction meeting. Worked with elicitation techniques and requirements elicitation, and written parts of the project experiences document. 

\newpage
%\addcontentsline{toc}{section}{References}
\begin{thebibliography}{1}
\bibitem{srs} Crash Software Requirements Specification version 1. 
\bibitem{pmv1} Crash Project Mission version 1.
\bibitem{pmv2} Crash Project Mission version 2.
\end{thebibliography}
\newpage

%%%%% APPENDIX %%%%%%%%%
\appendix
\section{Questions for interview 1}
Påståendena inom citationstecken är tagna from kundens PMv1.

\noindent”Följa rådande trafikregler”:
\begin{itemize}
\item Vilka länders trafikregler? (Kalifornien, Nevada, Michigan och Florida enda lagliga delstater i USA)
\item Ta hänsyn om regler för självstyrande bilar? Vilka finns det?
\item Scenario: En gren ligger på vägen och det finns inga andra bilar åt någon utav sidorna men där är en heldragen linje – Får man då byta körfält för att undvika krock eller ska man stanna och riskera krock bakifrån eller bildande av köer? (Finns det andra tillfällen då det är ok att bryta mot trafikregler?)
\item Får man köra snabbare än hastighetsgränsen vid nödfall?
\item Får man köra med t.ex. en trasig lampa om det är nödfall?
\item Vad händer om det är lågt däcktryck (säkerhetsrisk) men det är ett nödfall?
\end{itemize}
\hfill \break

\noindent ”Ta hänsyn till händelser i omgivningen”:
\begin{itemize}
\item Vilka typer av händelser? (se föregående fråga)
\end{itemize}
\hfill \break

\noindent ”Ta hänsyn till rådande trafiksituation”:
\begin{itemize}
\item Vad innebär detta?
\item Hur skiljer det sig från föregående två punkter?
\end{itemize}
\hfill \break

\noindent ”Framföra fordonet på ett komfortabelt vis för eventuella passagerare”:
\begin{itemize}
\item Vad innebär komfortabelt i det här sammanhanget? 
\item Vilka riktlinjer finns det?
\item Räknas förare in som passagerare?
\item Om bilen framförs utan förare/passagerare gäller samma regler för komfort då?
\end{itemize}
\hfill \break

\noindent ”Använda bränsle sparsamt”:
\begin{itemize}
\item Hur sparsamt? 
\item Vilka riktlinjer finns det?
\end{itemize}
\hfill \break

\noindent ”Ta hänsyn till väderomständigheter”:
\begin{itemize}
\item På vilket sätt ska man ta hänsyn till väderomständigheter?
\item Vilka väderomständigheter ska man ta hänsyn till?
\item När är det för osäkert för att köra?
\end{itemize}
\hfill \break

\noindent ”Ta hänsyn till återstående räckvidd”:
\begin{itemize}
\item På vilket sätt ska man ta hänsyn till återstående räckvidd?
\item "Köra 10 km/h för att komma till målet"?
\item Planera om rutt för bensinmack?
\item Vända/ta en omväg om bränslet inte räcker?
\end{itemize}
\hfill \break

\noindent ”Användaren ska kunna mata in destination”:
\begin{itemize}
\item Hur matar man in destination? (tangentbord/touch/voice)
\item Hur ska man få feedback?
\end{itemize}
\hfill \break

\noindent"Det ska finnas ett "föräldraläge", om man placerar sitt barn ensamt i bilen så ska barnet inte kunna ändra destination":
\begin{itemize}
\item Hur avancerat ska detta vara?
\item Vilka begränsningar finns det?
\item Är detta upp till oss att bestämma?
\end{itemize}
\hfill \break

\noindent ”Flera självkörande bilar ska kunna länka sig samman och forma ett sorts tåg, och på så vis kunna spara bränsle, och utnyttja vägnätet mer effektivt”:
\begin{itemize}
\item Spara hur mycket bränsle?
\item Förklara riktlinjerna för hur effektivt detta är.
\item Hur eftersträvansvärt är detta?
\item Får man köra om andra bilar för att komma ikapp en annan bil för att skapa tåg? (Problem om varannan bil är självstyrande isf)
\item Är det inte farligt om man inte släpper in vanliga bilar för att behålla tåget? 
\end{itemize}
\hfill \break

\noindent Övriga frågor:
\begin{itemize}
\item Vad finns det för prioriteringar mellan ovanstående krav? Går tex komfort eller bränsleförbrukning först?
\item Hur gör man med cyklister vid övergångsställen? (man ska ju ej stanna enligt trafikregler)
\item Vad finns det för hårdvarubegränsningar?
\item Vilken bränsletyp går bilen på?
\item Vad ska man ha för säkerhetsmarginaler?
\item Ska föraren kunna ingripa/köra bilen?
\item Hur gör man med ”manuella trafikljus”? (trafikpoliser och dylikt)
\item Hur gör man vid trasiga trafikljus (gul-blinkande), poliskontroller etc?
\item Ska den stödja alla typer av parkering?
\item Vad finns det för säkerhet för att öppna dörrar?
\item Vad ska bilen göra vid krock/skada?
\item Vad händer om bilen blir stulen?
\item Var mellan olika alternativ när krock är oundviklig? (Trolley problem)
\item Vilken hastighet ska bilen köra i när maxhastighet inte finns?
\item Vad händer om GPS inte är tillgänglig?
\item Ska bilen kunna släppa fram blåljus?
\end{itemize}

\newpage
\section{Mindmap of the system}

\begin{figure}[htb]    
 \centering
  \includegraphics[width=1\textwidth]
    {"Mindmap".png}% bild på vårat context diagram
  \caption{Mindmap of the system. Created in the beginning of the project.}
  \label{fi:Mindmap}
\end{figure}

\newpage
\section{First requirement classification}

\begin{figure}[htb]    
 \centering
  \includegraphics[width=1\textwidth]
    {"kravindelning".jpg}% bild på vårat context diagram
  \caption{Our first requirements classification.}
  \label{fi:classification}
\end{figure}

\end{document}
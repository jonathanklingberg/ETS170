%%%%%%%%%%%%%%%%%%%%%%%%%%%%%%%%%%%%%%%%%%%%%%%%%%%%%%%%%%%%%%%%%%%%%%
%
% If you're new to LaTeX, the wikibook is a great place to start:
% http://en.wikibooks.org/wiki/LaTeX
%
%%%%%%%%%%%%%%%%%%%%%%%%%%%%%%%%%%%%%%%%%%%%%%%%%%%%%%%%%%%%%%%%%%%%%%
%
% Template for PLoS
% Version 1.0 January 2009
%
% Edit the title below to update the display in My Documents
%\title{PLoS Journal Article}

\documentclass[10pt]{article}

% amsmath package, useful for mathematical formulas
\usepackage{amsmath}
% amssymb package, useful for mathematical symbols
\usepackage{amssymb}

% graphicx package, useful for including eps and pdf graphics
% include graphics with the command \includegraphics
\usepackage{graphicx}

% cite package, to clean up citations in the main text. Do not remove.
\usepackage{cite}

\usepackage{color} 

\usepackage[utf8]{inputenc}

\usepackage[none]{hyphenat}
\title{Project Experiences}
%\subtitle{(Subtitle)}
\author{CRASH Project
\\
\\ Grupp F}




\usepackage{titlesec}
\titlespacing\section{0pt}{12pt plus 4pt minus 2pt}{0pt plus 2pt minus 2pt}
\setcounter{secnumdepth}{5}
\setcounter{tocdepth}{5}

% Use doublespacing - comment out for single spacing
\usepackage{setspace} 

% Text layout
\topmargin 0.0cm
\oddsidemargin 0.5cm
\evensidemargin 0.5cm
\textwidth 16cm 
\textheight 21cm

% Bold the 'Figure #' in the caption and separate it with a period
% Captions will be left justified
\usepackage[labelfont=bf,labelsep=period,justification=raggedright]{caption}
\captionsetup[table]{name=Tabell}
% Use the PLoS provided bibtex style
\bibliographystyle{plos2009}

% Remove brackets from numbering in List of References
\makeatletter
\renewcommand{\@biblabel}[1]{\quad#1.}
\makeatother

% Made sure "contents" is in swedish
\renewcommand{\contentsname}{Innehållsförteckning}

% Remove page numbers
%\pagenumbering{gobble}


% Leave date blank
\date{}

\pagestyle{myheadings}
%% ** EDIT HERE **


%% ** EDIT HERE **
%% PLEASE INCLUDE ALL MACROS BELOW

%% END MACROS SECTION

\begin{document}

% Title must be 150 characters or less
\date{\today}
%\begin{center}
\begin{titlepage}
\clearpage
  \maketitle
\thispagestyle{empty}

\end{titlepage}

%{\Large
%\textbf{CRASH - Comfort, Reliability and Self Handling}
%}
%\end{center}

%\newpage
\tableofcontents
\thispagestyle{empty}
\newpage
\pagenumbering{arabic}
% Projektets bakgrund
\section{From the course description}
\sloppy
\noindent
Project Experiences including the following information:
\\(a) a description of your requirements engineering work, including experiences
and reflections in relation to learning objectives.
\\(b) Description of the chosen methods/techniques for elicitation, specifi-
cation, validation, and prioritization.
\\(c) Motivation for why you chose the used methods/techniques.
\\(d) Reflection on the usage of these methods/techniques in terms of what
was successful and what was challenging. Example questions for re-
flection: What have you learned in relation to the learning objectives
in this course program? What would you have done differently if you
would do this project again as a ”real” project, based on what you know
now? What have you learned in relation to the learning objectives?
\\(e) A personal statement by each team member that briefly explains each
individual’s contributions to the project results.
\\(f) The Project Experiences should not include course evaluation issues,
but focus on your own work and learning outcome.
\newline
\section{Template start here}
\newpage
\section{Requirement engineering work}
\sloppy
\noindent To get a great overview of the features we createad a mindmap where we very easily could cut offs branches to shrinken the project scope.

% Projektets mål   
\section{Elicitation}
\sloppy
\noindent Some description of elicitation.
\subsection{Elicitation barriers}
Some description of elicitation barriers.
\subsubsection{Customers changes their minds}
One of thing that was made clear after the first interview was that the system should focus on a private audience but soon after this interview they changed their minds and decided it should be possible to integrate into a taxi-activity for enterprise purposes.
\subsubsection{Luxury requirements}
During our first interview we felt that the customers came up with too many requirements. Some of them were essential, while others were just “nice-to-have”, or so called luxury requirements. For example, they wanted the possibility for the autonomous cars to form a train so that the road network would be used more efficiently. While this was a nice feature to have, we did not feel that it was a necessity for achieving their main goals. This and similar disagreements were solved by negotiations and reasoning. 
\subsubsection{Future requirements}
This really needed?
\subsection{Elicitation techniques}
For the first release of the requirements specification we decided to use the following elicitation techniques:
Besides these we’ve tried observation but we found no use of this, document study.
Task demostration kankse
Focus group should be of great benefits but we don’t have resourses enough for this.
Domain workshops aren’t applicable since there exists no expert users for this domain.
We’re going to use prototyping for our next release.
Main purpose of elicitation is to figure out our customer goals.

\subsubsection{Stakeholder analysis}
In a stakeholder analysis a research to find all the people who’s needed to ensure the success of the project is carried out. These people are then either interviewed or gathered to a joint meeting in order to get their ideas about the system. The goal of the meeting is to find answers to the following questions:
\begin{itemize}
\item What goals do they see for the system?
\item Why would they like to contribute?
\item What risks and costs do they see?
\item What kind of solutions and suppliers do they see?
\end{itemize}





\paragraph{Integration}
We listed all of the people and organizations that is needed to ensure the success of the project, the list can be found in reference [1]. All of the stakeholders have major intrests in enhanced traffic-security.
Motivation
It’s a simple way to summarize all partys that have interest in using this system and to get an overview of the people who probably will come in contact with this system.
Reflection
Most of our stakeholders are large organizations and don’t


% Projektets funktionalitet
\section{Funktionalitet}
\sloppy
\noindent Systemet ska
\begin{itemize}
	\setlength\itemsep{0.1em}
	\item Alltid prioritera passagerarnas och omgivningens säkerhet högst.
	\item När en bilolycka är oundviklig ska styrsystemet agera på ett sådant sätt att skador minimeras och räddande av människoliv prioriteras.
	\item Följa svenska trafikregler.
	\item Ta hänsyn till händelser i sin omgivning och rådande trafiksitution  när aktuell rutt bestäms.
	\item Framföra fordonet på ett för passageraren komfortabelt sätt.
	\item Hantera bränsle sparsamt.
	\item Ta hänsyn till rådande och förutspådda prognoser om framtida väderomständigheter vid säkerhetsvärdering.
	\item Ta hänsyn till bilens skick utifrån sensordata vid säkerhetsvärdering.
	\item Ta hänsyn till återstående räckvidd samt kunna lägga om rutten för att tanka vid behov.
	\item Vid konflikt prioritera i följande ordning; säkerhet, trafikregler, komfort och bränsleeffektivitet.
	\item Autentisera förare med hjälp av en fingeravtryckssensor.
	\item Stödja inmatning av destination genom röststyrning eller en tillgänglig pekskärm på bilens instrumentpanel samt ge feedback på detta både med ljudmeddelanden och visuellt.
	\item Kunna framföras utan passagerare.
	\item Tillåta att bilen körs manuellt om detta läge valts.
\end{itemize}


% Vår roll i projektet
\section{Roller}
\sloppy
\noindent LADA har konstaterat att kompetensen för att utveckla kravspecifikationen inte finns internt i företaget, av denna anledning har det fattats ett beslut om att kravspecifikationen ska  förvärvas externt och att LADA ska ta rollen som nyckelkund.

\subsection{Kundteam: grupp I}
\noindent
Johan Barkfors, zba10jan@student.lu.se (SCCVM)
\\Olof Spångö, ain09osp@student.lu.se
\\Daniel Jigin, elt11dji@student.lu.se
\\Max Andersson, elt11ma1@student.lu.se
\\Jacob Hedqvist, elt11jhe@student.lu.se
\\Andreas Wiberg, elt11awi@student.lu.se (P3RM)
\\Mattias Mellhorn, jcd11mto@student.lu.se


\subsection{Utvecklingsteam: grupp F}
\noindent
Alexander Badju, adi10aba@student.lu.se (P3RM)	
\\*Fredrik Helander, gda10fhe@student.lu.se (EPM)
\\*Jonathan Klingberg, adi10jkl@student.lu.se (SCCVM/TDEVM)
\\*Jonathan Knorn, ada09jkn@student.lu.se (TDEVM)
\\*David Lundberg, adi10dlu@student.lu.se (QRM)
\\*Niklas Sjöberg, adi10nsj@student.lu.se (DRM)

\section{Potentiella intressenter}
\sloppy
\noindent
Det finns potentiellt ett stort antal intressenter i det här projektet. Vi har valt att fokusera på de enligt oss största och viktigaste inre, respektive yttre intressenterna.




\end{document}

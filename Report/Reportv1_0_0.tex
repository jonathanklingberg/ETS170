%%%%%%%%%%%%%%%%%%%%%%%%%%%%%%%%%%%%%%%%%%%%%%%%%%%%%%%%%%%%%%%%%%%%%%
%
% If you're new to LaTeX, the wikibook is a great place to start:
% http://en.wikibooks.org/wiki/LaTeX
%
%%%%%%%%%%%%%%%%%%%%%%%%%%%%%%%%%%%%%%%%%%%%%%%%%%%%%%%%%%%%%%%%%%%%%%
%
% Template for PLoS
% Version 1.0 January 2009
%
% Edit the title below to update the display in My Documents
%\title{PLoS Journal Article}

\documentclass[10pt]{article}

% amsmath package, useful for mathematical formulas
\usepackage{amsmath}
% amssymb package, useful for mathematical symbols
\usepackage{amssymb}

% graphicx package, useful for including eps and pdf graphics
% include graphics with the command \includegraphics
\usepackage{graphicx}

% cite package, to clean up citations in the main text. Do not remove.
\usepackage{cite}

\usepackage{color} 

\usepackage[utf8]{inputenc}

\usepackage[none]{hyphenat}
\title{Project Experiences}
%\subtitle{(Subtitle)}
\author{CRASH Project
\\
\\ Grupp F}


\usepackage[nottoc,notlot,notlof]{tocbibind}

\usepackage{titlesec}
\titlespacing\section{0pt}{12pt plus 4pt minus 2pt}{0pt plus 2pt minus 2pt}
\setcounter{secnumdepth}{5}
\setcounter{tocdepth}{5}

% Use doublespacing - comment out for single spacing
\usepackage{setspace} 

% Text layout
\topmargin 0.0cm
\oddsidemargin 0.5cm
\evensidemargin 0.5cm
\textwidth 16cm 
\textheight 21cm

% Bold the 'Figure #' in the caption and separate it with a period
% Captions will be left justified
\usepackage[labelfont=bf,labelsep=period,justification=raggedright]{caption}
\captionsetup[table]{name=Tabell}
% Use the PLoS provided bibtex style
\bibliographystyle{plos2009}


% Remove brackets from numbering in List of References
\makeatletter
\renewcommand{\@biblabel}[1]{\quad#1.}
\makeatother

% Made sure "contents" is in swedish
\renewcommand{\contentsname}{Innehållsförteckning}

% Remove page numbers
%\pagenumbering{gobble}


% Leave date blank
\date{}

\pagestyle{myheadings}
%% ** EDIT HERE **


%% ** EDIT HERE **
%% PLEASE INCLUDE ALL MACROS BELOW

%% END MACROS SECTION

\begin{document}

% Title must be 150 characters or less
\date{\today}
%\begin{center}
\begin{titlepage}
\clearpage
  \maketitle
\thispagestyle{empty}

\end{titlepage}

%{\Large
%\textbf{CRASH - Comfort, Reliability and Self Handling}
%}
%\end{center}

%\newpage
\tableofcontents
\thispagestyle{empty}
\newpage
\pagenumbering{arabic}

\section{From the course description}
\sloppy
\noindent
Project Experiences including the following information:
\\(a) a description of your requirements engineering work, including experiences
and reflections in relation to learning objectives.
\\(b) Description of the chosen methods/techniques for elicitation, specifi-
cation, validation, and prioritization.
\\(c) Motivation for why you chose the used methods/techniques.
\\(d) Reflection on the usage of these methods/techniques in terms of what
was successful and what was challenging. Example questions for re-
flection: What have you learned in relation to the learning objectives
in this course program? What would you have done differently if you
would do this project again as a ”real” project, based on what you know
now? What have you learned in relation to the learning objectives?
\\(e) A personal statement by each team member that briefly explains each
individual’s contributions to the project results.
\\(f) The Project Experiences should not include course evaluation issues,
but focus on your own work and learning outcome.
\newline

%%%%% Requirement engineering work %%%%%%%%% 
\section{Requirement engineering work}
\noindent For this project we'll have three releases which project experiences and main objectives are described below.
\subsection{Release 1}
The project started off by us setting up a group interview with the customers where they were allowed  to explain all of their thought and ideas without we even questioning them. After this meeting we realized the project scope would be too large to be able to handle within the scope of this course and needed to focus on the most vital functionalities.
To get a first overview of the system features we started by creating a mind map where we could cut off branches not vital for the project and to reduce the project scope very easily.
The main objective for release 1 was to focus on elicitation techniques to get a grip of the system requirements.

%%%%% Elicitation %%%%%%%%% 
\section{Elicitation}
\sloppy
\noindent Some description of elicitation.
\subsection{Elicitation barriers}
The purpose of a system requirement and specification
Elicitation barriers
During the project elicitation phrase we experienced some elicitation barriers described below. 
\subsubsection{Customers changes their minds}
One of thing that was made clear after the first interview was that the system should focus on a private audience but soon after this interview they changed their minds and decided it should be possible to integrate into a taxi-activity for enterprise purposes.
\subsubsection{Luxury requirements}
During our first interview we felt that the customers came up with too many requirements. Some of them were essential, while others were just “nice-to-have”, or so called luxury requirements. For example, they wanted the possibility for the autonomous cars to form a train so that the road network would be used more efficiently. While this was a nice feature to have, we did not feel that it was a necessity for achieving their main goals. This and similar disagreements were solved by negotiations and reasoning. 

\subsection{Elicitation techniques}
For the first release of the requirements specification we decided to use the following elicitation techniques:
Besides these we’ve tried observation but we found no use of this, document study.
Task demostration kankse
Focus group should be of great benefits but we don’t have resourses enough for this.
Domain workshops aren’t applicable since there exists no expert users for this domain.
We’re going to use prototyping for our next release.
Main purpose of elicitation is to figure out our customer goals.

\subsubsection{Stakeholder analysis}
In a stakeholder analysis a research to find all the people who’s needed to ensure the success of the project is carried out. These people are then either interviewed or gathered to a joint meeting in order to get their ideas about the system. The goal of the meeting is to find answers to the following questions:
\begin{itemize}
\item What goals do they see for the system?
\item Why would they like to contribute?
\item What risks and costs do they see?
\item What kind of solutions and suppliers do they see?
\end{itemize}

\paragraph{Execution}
We listed all of the people and organizations that is needed to ensure the success of the project, the list can be found in reference \cite{pmv2}. All of the stakeholders have major interests in enhanced traffic-security.
\paragraph{Motivation}
It’s a simple way to summarize all partys that have interest in using this system and to get an overview of the people who probably will come in contact with this system.
\paragraph{Reflection}
Most of our stakeholders are large organizations and don’t


\subsubsection{Interviewing}
Interviewing is one of the most widely used eliciting techniques, but it has its pros and cons. It is a quite simple and straightforward technique that requires little planning and can be used in various situations. It is a good technique for getting information about the present work in the domain as well as present problems. It can also help identify where conflicts may lie, however other techniques are needed to resolve the conflicts (for example we used negotiation as a technique for resolving conflicts with our stakeholders, this will be discussed later in the report). Interviewing is not as good at identifying the goals and critical issues.
When doing interviews it is important, or at least preferred, that you ask members from each user group. Management often choose to officially nominate representatives for a user group. However, experience have shown that representatives generally don’t know what is going on in the daily business. Therefore, it is recommended to interview other staff members as well. Interviews can be conducted with either individuals or in groups.
Generally there are two types of interviews: 
Structured -  the interviewer has prepared a set of questions that needs to be answered. 
Unstructured - the interviewer has not prepared any questions and instead openly discuss what is expected from the system. 
\paragraph{Execution}
We felt that a structured interview where we prepared a list of questions to ask where most fitting when meeting with our customers. Unstructured interviews requires more experience and are harder to perform. However, we did leave room for follow-up questions to confirm our understanding of what the customer asked for and why they asked for it. 
\paragraph{Motivation}
Since the Project Mission where quite vague and left out a lot of functionality (intentionally), we had a lot of questions about the system that needed to be answered before we could start working on the requirements specification. We felt that we needed to get clarification fairly soon, and as interviewing requires little planning and work, and doesn’t take to long to perform, it was deemed a fitting technique to use. Interviews also allows the interviewer and interviewee(s) to discuss and explain the questions and answers, which can give even further clarification. It also gives us the possibility to ask follow-up questions to confirm our understanding. 
\paragraph{Reflection}
What to put here??????

\subsubsection{Brainstorming}
Using brainstorming as an elicitation technique is an effective way of generating new ideas. During a brainstorm session it is of great importance not to criticize any idea. This creates an environment where no group member is afraid of being ridiculed and thus enable the group produce new ideas at a higher rate.
The process of separating good ideas from bad ones is saved until after the brainstorming session. 
\paragraph{Execution}
We held brainstorming sessions i a room where we had access to a whiteboard. During the sessions we wrote down the ideas and requirements we came up with on the board.
\paragraph{Motivation}
We decided to use this elicitation technique as it was a natural step in the beginning of the elicitation process. The technique enables us to quickly gather a set of ideas on how to handle new problems that occur.
\paragraph{Reflection}
The technique allowed us not only to come up with new requirements but also enabled us, in the beginning of the elicitation process, to get an overview of the system we are building.


\subsubsection{Negotiation}
\paragraph{Execution}
\paragraph{Motivation}
\paragraph{Reflection}


\subsubsection{Domain-requirements analysis}
\paragraph{Execution}
\paragraph{Motivation}
\paragraph{Reflection}


\subsubsection{Similar companies}
\paragraph{Execution}
\paragraph{Motivation}
\paragraph{Reflection}


\subsubsection{Goal-domain analysis}
\paragraph{Execution}
\paragraph{Motivation}
\paragraph{Reflection}

%%%%% INDIVIDUAL STATEMENTS %%%%%%%%%
\section{Individual statements}
\noindent
\subsection{Alexander Badju, P3RM}
As project leader my contribution to this weeks work has consisted of organizing some of the work load. Besides the regular group work, such as creating the context diagram etc. I have mostly worked with developing system requirements as well as scenarios. 
\subsection{Fredrik Helander, EPM}
\subsection{Jonathan Klingberg, SCCVM/TDEVM}
Besides the regular project work such as meetings and group discussions, Jonathan has also been working with setting up tools and some initial tool guidance for the other project members.
\subsection{Jonathan Knorn, TDEVM}
\subsection{David Lundberg, QRM}
\subsection{Niklas Sjöberg, DRM}

%\addcontentsline{toc}{section}{References}
\begin{thebibliography}{1}
\bibitem{pmv2} Project Mission version 2. 

\end{thebibliography}

\end{document}

\documentclass{article}
\usepackage[utf8]{inputenc}
\usepackage{titling}
\usepackage{graphicx}
\usepackage{sidecap} % SCfigur
\usepackage[compact,explicit]{titlesec}
\usepackage{hyperref}
\usepackage{todonotes}


%KNORN
\usepackage{hyperref}
\hypersetup{colorlinks=true, linkcolor=blue, urlcolor=blue}
%\usepackage[usenames,dvipsnames,svgnames,table]{xcolor}
\definecolor{entityColor}{RGB}{0,100,200}
\definecolor{attributeColor}{RGB}{0,100,50}
\definecolor{relationColor}{RGB}{160,0,30}
\usepackage{listings}
\lstdefinestyle{reqT}{
  belowcaptionskip=1\baselineskip,
  breaklines=true,
  showstringspaces=false,
  basicstyle=\footnotesize\sffamily,
  emph={Ent,Meta,Item,Label,Section,Term,Actor,App,Component,Domain,Module,Product,Release,Resource,Risk,Service,Stakeholder,System,User,Class,Data,Input,Member,Output,Relationship,Design,Screen,MockUp,Function,Interface,Epic,Feature,Goal,Idea,Issue,Req,Ticket,WorkPackage,Breakpoint,Barrier,Quality,Target,Scenario,Task,Test,Story,UseCase,VariationPoint,Variant},
  emphstyle=\bfseries\color{entityColor},
  emph={[2]has,is,superOf,binds,deprecates,excludes,helps,hurts,impacts,implements,interactsWith,precedes,requires,relatesTo,verifies},
  emphstyle={[2]\bfseries\color{relationColor}},
  emph={[3]Attr,Code,Constraints,Comment,Deprecated,Example,Expectation,FileName,Gist,Image,Spec,Text,Title,Why,Benefit,Capacity,Cost,Damage,Frequency,Min,Max,Order,Prio,Probability,Profit,Value,Status},
  emphstyle={[3]\color{attributeColor}},  
}
\lstset{style=reqT}

%KNORN

\begin{document}

% Raden nedan ser till att nummret på en subsubsection kommer EFTER dess titel.
% Det är nuvarande lösning för att få kraven på formatet "Requirement 6.1.3"
\titleformat{\subsubsection}[runin]{\large\bfseries}{}{0pt}{#1\quad\thesubsubsection}

\begin{titlepage}

\noindent
\textsc{} \\ % The group responsible for the document, such as Development Group
\textsc{Alexander Badju, Fredrik Helander, Jonathan Klingberg, Jonathan Knorn, David Lundberg & Niklas Sjöberg}\\ % Names of the authors of the document 
\textsc{24/11-14}\\ % The date
\textsc{}\\ % The document number, eg PUSS124819. See page 22 in the "Projekthandledning"
\textsc{} \\ % The document version, such as 0.1 or 1.0

\vspace{10em}
\begin{flushright}

{\huge Software Requirements Specification R1}\\ % The document name

{\large Group F - Steve's Angels} \\
\par\end{flushright}\vskip 0.5em
\end{titlepage}

\newpage

\section*{Document history}

\begin{center}
    \begin{tabular}{ l l l l }
      Version       &      Date             &       Responsible    &   Description     \\ \hline
      1.0.0 & \date{\today}  & Fredrik Helander & Initial specification for release 1
      
    \end{tabular}
\end{center}

\newpage

\tableofcontents

\newpage

\section{Introduction}
The car manufacturer LADA is planning on taking market shares through their new product, a selfdriving car. LADA themselves do not feel that they can produce a requirement specification to the quality standard they feel is needed, which is why they have hired our company, Steve's Angels, to produce the requirements specification. 

\section{Reference documents} % change to normal list to look like SVVS?
\begin{enumerate}
\item CRASH - Comfort, Reliability and Self Handling, Project Mission v2
\end{enumerate}

\section{Background and goals}
\subsection{Main goals}
The main goal with this document is to provide the first release of system requirements to our customer LADA for their autonomous car system they want us to develop the requirement specification. 


\section{Actors and their objectives}


%\begin{figure}[htb]      Om vi ska ha en bild
 % \centering
  %\includegraphics[width=0.9\textwidth]%
   % {userHierarchy.jpg}% picture filename
  %\caption{Showing the user hierarchy of the system.}
  %\label{hierarchy}
%\end{figure}

\textbf{4.1 Passenger} There are two types of passengers, a passive passenger or a authenticated passenger. Both can press the emergency stop button, but only the authenticated passenger can ride in the car alone.\\
\textbf{4.2 Driver} Can do anything that a passenger can.
The driver can also drive the car manually, handle the voice control as well as choosing the destination. \\
\textbf{4.3 Owner} Can do anything that a driver can. The owner also handles user management. Which users are allowed to be drivers and passengers. \\
\textbf{4.4 Administrator} Can do anything that a owner can
Change system settings as well as general admin management \\





%\begin{figure}[htb]    En bild till
 % \centering
  %\includegraphics[width=0.9\textwidth]%
   % {context_diagram1.png}% picture filename
 % \caption{Context diagram showing actors and key components within the system.}
  %\label{ContextDiagram}
%\end{figure}

\section{Terminology}
\textbf{Autonomous car} - An autonomous car is also known as a self driving car. \\
\textbf{The system} - The delivered product. When nothing else is stated, requirements are specified for autonomous driving mode.\\
\textbf{Ecodriving} - Ecodriving is a term used to describe energy efficient use of vehicles. It is a great and easy way to reduce fuel consumption from road transport so that less fuel is used to travel the same distance.
  \\
\textbf{Unsafe state} - The unsafe state is the cars state when it does not fulfill Swedish car inspection regulations.\\


\section{Context diagram}

\begin{figure}[htb]    
 \centering
  \includegraphics[width=0.9\textwidth]%
    {"Context diagram v4 - New Page".png}% bild på vårat context diagram
  \caption{Context diagram showing actors and key components within the system.}
  \label{ContextDiagram}
\end{figure}

A context diagram of the system can be seen in Figure \ref{ContextDiagram}.

\section{Dependencies}

\begin{lstlisting}
Domain FDo1 requires
  Product DPr1
  Domain FDo4
  Domain FDo7
  Product FPr11
Domain FDo2 requires
  Domain FDo5
Domain FDo3 requires
  Product FPr11
Domain FDo4a requires
  Domain FDo5
Domain FDo4b requires
  Domain FDo5
Domain FDo12 requires
  Domain FDo7
Domain FDo15a requires
  Domain DDo1a
  Domain DDo1b
  Domain DDo1c
  Domain DDo1d
Domain FDo15b requires
  Domain DDo1a
  Domain DDo1b
  Domain DDo1c
  Domain DDo1d
Product FPr1 requires
  Domain FDo6
Product FPr2 requires
  Product DPr3
  Product FPr1
Product FPr3 requires
  Domain FDo6
Product FPr6 requires
  Product FPr5
Product FPr7 requires
  Domain FDo8
Product FPr9 requires
  Domain FDo10
Product FPr11 requires
  Product DPr6
Domain DDo2 requires
  Product DPr3
Domain DDo4 requires
  Domain DDo3
Product DPr7 requires
  Product Dpr2
Domain QDo1 requires
  Domain FDo5
Product QPr1 requires
  Domain FDo6
Product QPr2 requires
  Domain FDo6
Product QPr3 requires
  Domain FDo5
Product QPr4 requires
  Domain FDo6
Product FPr8 requires
  Domain FDo1
  Domain FDo8
Domain DDo3 requires
  Product Dpr2
  Product DPr3
Product FPr12 requires
  Domain FDo5
  Product FPr11

\end{lstlisting}


       \section{Functional requirements}


\begin{lstlisting}

\end{lstlisting}


       \subsection{Functional goal requirements}


\begin{lstlisting}
Goal FGo1
  Spec Autonomous car

\end{lstlisting}
    
        
       \subsection{Functional domain requirements}


\begin{lstlisting}
Domain FDo1
  Spec The system must adhere to Swedish traffic laws
Domain FDo2
  Spec If an accident can be avoided, the system may break traffic laws.
Domain FDo3
  Spec When the car is in an unsafe state the car must not be able/allowed to drive.
Domain FDo4a
  Spec In the situation of an accident, the system must prioritize risks in the following order: Saving as many human lives as possible has top priority
Domain FDo4b
  Spec In the situation of an accident, the system must prioritize risks in the following order: Protecting humans inside the car is prioritized over humans outside the car
Domain FDo5
  Spec The system requires adequate sensors for monitoring the road and surroundings.
Domain FDo6
  Spec The system requires an Internet connection
Domain FDo7
  Spec The system requires a GPS-sensor
Domain FDo8
  Spec The system requires a breathalyzer.
Domain FDo9
  Spec The system requires a fingerprint-sensor.
Domain FDo10
  Spec The system requires a microphone.
Domain FDo11
  Spec The system functions without a GPS-signal once a route has been chosen.
Domain FDo12
  Spec The system needs GPS-signal at startup.
Domain FDo13
  Spec The system must strive to ecodrive.
Domain FDo14
  Spec All user interaction that is available in the system must be possible to enter remotely.
Domain FDo15a
  Spec The system must support: permanent user rights
Domain FDo15b
  Spec The system must support: temporary user rights

\end{lstlisting}
    
        
       \subsection{Functional product requirements}


\begin{lstlisting}
Product FPr1
  Spec The system has to to retrieve current and future weather data using from the Internet when a  connection is available.
Product FPr2
  Spec The system must give a warning when dangerous weather conditions are predicted along the planned route.
Product FPr3
  Spec The system has to retrieve data of current and future traffic situations when an Internet connection is available.
Product FPr4
  Spec The system must always be able to stop via an emergency-break.
Product FPr5
  Spec It must always be possible to request to turn off the autonomous system and drive the car manually.
Product FPr6
  Spec When manual driving has been requested, the car has to be standing still before the autonomous system is turned off.
Product FPr7
  Spec For a passenger to be able to turn off the system and drive the car manually, the person has to blow under the legal limit in the breathalyzer.
Product FPr8
  Spec The breathalyzer must be configured according to Swedish traffic law
Product FPr9
  Spec The system must support voice controlled input
Product FPr10
  Spec The system must input from the car's dashboard
Product FPr11
  Spec The system has to be able to evaluate the car's status compared to current Swedish car inspection rules before driving off.
Product FPr12
  Spec The system must order a towing service when it is in an unsafe state.

\end{lstlisting}
    
        
       \subsection{Functional scenario requirements}


\begin{lstlisting}
Scenario FSc1
Spec  Niklas Sjöberg’s football practice is over and he remotely tells his car to pick him and his friends up at the football pitch, he uses his phone and the CRASH-application. The next step is that the car drives to pick up Niklas and his friends. When the car arrives, Niklas authenticates himself by fingerprint with the cars fingerprint scanner located on the car key. When the authentication is done, Niklas and his friends enter the car and Niklas tells the car where to drive using voice control, when this is done, the car drives the passengers to the given addresses. One of Niklas’ friends lives further away than Niklas himself and he is tired and he wants to be dropped off first. Since Niklas was the driver and now leaves the car, the passenger left in the car has to authenticate himself and thus becomes a authenticated passenger, using the fingerprint scanner.
Scenario FSc2
  Spec Jonathan Klingberg is at work and something very important has just come up, which means that he cannot pick up his child at school as planned. Instead he uses his phone to remotely tell his car to pick up his child. The child receives a notification from the car that it is on its way for a pick up and how long it should take to arrive. When the car arrives, the child authenticates itself with the cars built-in fingerprint scanner and enters the car. The car drives the child home with the fathers predetermined route input.
  
\noindent\textbf{Scenario FSc3} Successful car driver creation  \\
Precondition: Logged in as the owner on the touch panel in the car.
\begin{enumerate}
\item The administrator navigates to the users settings in the system preferences.
\iten The administrator selects the option for adding a new system user.
\item The system asks for user data of the new user.
\item The administrator enters the information which fulfills DPr2a, DPr2b, DPr2c.
\item The administrator chooses to submit the information.
\item The system responds with a success message and returns the administrator to an updated version of the users page.
\end{enumerate}

\end{lstlisting}
    
        
       \section{Data requirements}


\begin{lstlisting}

\end{lstlisting}


       \subsection{Data domain requirements}


\begin{lstlisting}
Domain DDo1a
  Spec The system must support the following authorized user types: Passenger
Domain DDo1b
  Spec The system must support the following authorized user types: Driver
Domain DDo1c
  Spec The system must support the following authorized user types: Owner
Domain DDo1d
  Spec The system must support the following authorized user types: Admin
Domain DDo2
  Spec A driver must be able to choose & change destination
Domain DDo3
  Spec An owner must be able to choose & change destination as well as handle users
Domain DDo4
  Spec An admin must have the same rights as an owner and the admin must be able to change system settings.

\end{lstlisting}
    
        
       \subsection{Data product requirements}


\begin{lstlisting}
Product DPr1
  Spec The system must store Swedish traffic rules.
Product DPr2a
  Spec The system must store the following user data: Name
Product DPr2b
  Spec The system must store the following user data: User type
Product DPr2c
  Spec The system must store the following user data: Fingerprint
Product DPr3
  Spec The system must store route data
Product DPr4
  Spec The system must store a service log
Product DPr5
  Spec The system must be able to store future traffic laws.
Product DPr6
  Spec The system must store Swedish car inspections rules and regulations.
Product DPr7
  Spec If there are users in the car, at least one needs to be authenticated.

\end{lstlisting}
    
        
       \section{Quality requirements}


\begin{lstlisting}

\end{lstlisting}


       \subsection{Quality goal requirements}


\begin{lstlisting}
Goal QGo1
  Spec The system must strive for maximal safety
Goal QGo2
  Spec The system response time must be lower than 100 ms

\end{lstlisting}
    
        
       \subsection{Quality domain requirements}


\begin{lstlisting}
Domain QDo1
  Spec The system is able to drive until the next connecting road without a GPS-signal

\end{lstlisting}
    
        
       \subsection{Quality product requirements}


\begin{lstlisting}
Product QPr1
  Spec The system must query for new traffic laws/regulations at least once a day.
Product QPr2
  Spec The system must query for new car inspection rules/regulations online at least once a day.
Product QPr3
  Spec System updates must only be performed when the car is parked.
Product QPr4
  Spec Traffic and weather data must be fetched every minute when an Internet connection is available

\end{lstlisting}

\end{document}

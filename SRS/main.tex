\documentclass{article}
\usepackage[utf8]{inputenc}
\usepackage{titling}
\usepackage{graphicx}
\usepackage{sidecap} % SCfigur
\usepackage[compact,explicit]{titlesec}
\usepackage{hyperref}
\usepackage{todonotes}


%KNORN
\usepackage{hyperref}
\hypersetup{colorlinks=true, linkcolor=blue, urlcolor=blue}
%\usepackage[usenames,dvipsnames,svgnames,table]{xcolor}
\definecolor{entityColor}{RGB}{0,100,200}
\definecolor{attributeColor}{RGB}{0,100,50}
\definecolor{relationColor}{RGB}{160,0,30}
\usepackage{listings}
\lstdefinestyle{reqT}{
  belowcaptionskip=1\baselineskip,
  breaklines=true,
  showstringspaces=false,
  basicstyle=\footnotesize\sffamily,
  emph={Ent,Meta,Item,Label,Section,Term,Actor,App,Component,Domain,Module,Product,Release,Resource,Risk,Service,Stakeholder,System,User,Class,Data,Input,Member,Output,Relationship,Design,Screen,MockUp,Function,Interface,Epic,Feature,Goal,Idea,Issue,Req,Ticket,WorkPackage,Breakpoint,Barrier,Quality,Target,Scenario,Task,Test,Story,UseCase,VariationPoint,Variant},
  emphstyle=\bfseries\color{entityColor},
  emph={[2]has,is,superOf,binds,deprecates,excludes,helps,hurts,impacts,implements,interactsWith,precedes,requires,relatesTo,verifies},
  emphstyle={[2]\bfseries\color{relationColor}},
  emph={[3]Attr,Code,Constraints,Comment,Deprecated,Example,Expectation,FileName,Gist,Image,Spec,Text,Title,Why,Benefit,Capacity,Cost,Damage,Frequency,Min,Max,Order,Prio,Probability,Profit,Value,Status},
  emphstyle={[3]\color{attributeColor}},  
}
\lstset{style=reqT}

%KNORN

\begin{document}

% Raden nedan ser till att nummret på en subsubsection kommer EFTER dess titel.
% Det är nuvarande lösning för att få kraven på formatet "Requirement 6.1.3"
\titleformat{\subsubsection}[runin]{\large\bfseries}{}{0pt}{#1\quad\thesubsubsection}

\begin{titlepage}

\noindent
\textsc{} \\ % The group responsible for the document, such as Development Group
\textsc{Alexander Badju, Fredrik Helander, Jonathan Klingberg, Jonathan Knorn, David Lundberg & Niklas Sjöberg}\\ % Names of the authors of the document 
\textsc{24/11-14}\\ % The date
\textsc{}\\ % The document number, eg PUSS124819. See page 22 in the "Projekthandledning"
\textsc{} \\ % The document version, such as 0.1 or 1.0

\vspace{10em}
\begin{flushright}

{\huge Software Requirements Specification R1}\\ % The document name

{\large Group F - Steve's Angels} \\
\par\end{flushright}\vskip 0.5em
\end{titlepage}

\newpage

\section*{Document history}

\begin{center}
    \begin{tabular}{ l l l l }
      Version       &      Date             &       Responsible    &   Description     \\ \hline
      1.0.0 & \date{\today}  & Fredrik Helander & Initial specification for release 1
      
    \end{tabular}
\end{center}

\newpage

\tableofcontents

\newpage

\section{Introduction}
The car manufacturer LADA is planning on taking market shares through their new product, a selfdriving car. LADA themselves do not feel that they can produce a requirement specification to the quality standard they feel is needed, which is why they have hired our company, Steve's Angels, to produce the requirements specification. 

\section{Reference documents} % change to normal list to look like SVVS?
\begin{enumerate}
\item CRASH - Comfort, Reliability and Self Handling, Project Mission v2
\end{enumerate}

\section{Background and goals}
\subsection{Main goals}
The main goal with this document is to provide the first release of system requirements to our customer LADA for their autonomous car system they want us to develop the requirement specification. 


\section{Actors and their objectives}


%\begin{figure}[htb]      Om vi ska ha en bild
 % \centering
  %\includegraphics[width=0.9\textwidth]%
   % {userHierarchy.jpg}% picture filename
  %\caption{Showing the user hierarchy of the system.}
  %\label{hierarchy}
%\end{figure}

\textbf{4.1 Passenger} There are two types of passengers, a passive passenger or a authenticated passenger. Both can press the emergency stop button, but only the authenticated passenger can ride in the car alone.\\
\textbf{4.2 Driver} Can do anything that a passenger can.
The driver can also drive the car manually, handle the voice control as well as choosing the destination. \\
\textbf{4.3 Owner} Can do anything that a driver can. The owner also handles user management. Which users are allowed to be drivers and passengers. \\
\textbf{4.4 Administrator} Can do anything that a owner can
Change system settings as well as general admin management \\





%\begin{figure}[htb]    En bild till
 % \centering
  %\includegraphics[width=0.9\textwidth]%
   % {context_diagram1.png}% picture filename
 % \caption{Context diagram showing actors and key components within the system.}
  %\label{ContextDiagram}
%\end{figure}

\section{Terminology}
\textbf{Autonomous car} - An autonomous car is also known as a self driving car. \\
\textbf{The system} - The delivered product. When nothing else is stated, requirements are specified for autonomous driving mode.\\
\textbf{Ecodriving} - Ecodriving is a term used to describe energy efficient use of vehicles. It is a great and easy way to reduce fuel consumption from road transport so that less fuel is used to travel the same distance.
  \\
\textbf{Unsafe state} - The unsafe state is the cars state when it does not fulfill Swedish car inspection regulations.\\


\section{Context diagram}

\begin{figure}[htb]    
 \centering
  \includegraphics[width=0.9\textwidth]%
    {"Context diagram v4 - New Page".png}% bild på vårat context diagram
  \caption{Context diagram showing actors and key components within the system.}
  \label{ContextDiagram}
\end{figure}

A context diagram of the system can be seen in Figure \ref{ContextDiagram}.

\begin{lstlisting}

\end{lstlisting}


       \section{GoalRequirements}


\begin{lstlisting}
Goal TEST
  Spec Detta ska funka enl. Jonte: \ref{srs}
Goal FGo1
  Spec Autonomous car for Swedish traffic
Goal FGo2
  Spec Reduce amount of traffic accidents on Swedish roads
Goal FGo3
  Spec Maintain industry market shares
Goal FGo4
  Spec Achieve maximal usability
Goal FGo5
  Spec Avoid human deaths in traffic
Goal FGo6
  Spec Achieve maximal security
Goal FGo7
  Spec Expand to other markets than private use
Goal FGo8
  Spec Achieve maximum comfort
Goal FGo9
  Spec Reduce cars' negative impacts on the environment
Goal QGo1
  Spec The system must strive for maximal safety
Goal QGo2
  Spec The system response time must be lower than 100 ms

\end{lstlisting}
    
        
       \section{ProductRequirements}


\begin{lstlisting}

\end{lstlisting}


       \subsection{FunctionalProductRequirements}


\begin{lstlisting}
Product FPr1
  Spec The system has to to retrieve current and future weather data using from the Internet when a connection is available
Product FPr2
  Spec The system must give a warning when dangerous weather conditions are predicted along the planned route
Product FPr3
  Spec The system has to retrieve data of current and future traffic situations when an Internet connection is available
Product FPr4
  Spec The system must always be possible to stop via an emergency-break
Product FPr5
  Spec It must always be possible to request to turn off the autonomous system and drive the car manually
Product FPr6
  Spec When manual driving has been requested, the car has to be standing still before the autonomous system is turned off
Product FPr7
  Spec For a passenger to be able to turn off the system and drive the car manually, the person has to blow under the legal limit in the breathalyzer
Product FPr8
  Spec The breathalyzer has to be configured according to Swedish traffic law
Product FPr9
  Spec The system has to support voice controlled input
Product FPr10
  Spec The system has to support input from the car's dashboard
Product FPr11
  Spec The system has to be able to evaluate the car's status compared to current Swedish car inspection rules before driving off
Product FPr12
  Spec The system has to order a towing service when it is in an unsafe state
Product FPr13
  Spec When the system is in manual driving mode, the system must still be active and avoid accidents the same way as in autonomous mode
Product FPr14
  Spec When a command is entered feedback must be provided to the user
Product FPr15
  Spec The car must evaluate the amount of fuel left in the tank
Product FPr16
  Spec When the fuel level in the car reaches the level where it cannot make it to the second nearest gas station along the route, the car must drive to the nearest gas station and reload with fuel
Product FPr17
  Spec If the voice control system cannot interpret an incomming voice command, it must suggest to the user what it interpreted
Product FPr18
  Spec The system must be able to drive with its sensor when the GPS and maps don't align with the reality
Product FPr19
  Spec When the emergency break is activated the system must stop completely at the earliest possible place without risking an accident and stay still until a new command is provided by an authenticated user
Product FPr20
  Spec The voice control system must support Swedish and English

\end{lstlisting}
    
        
       \subsection{DataProductRequirements}


\begin{lstlisting}
Product DPr1
  Spec The system must store Swedish traffic rules
Product DPr2
  Product DPr2a
    Spec The system must store the following user data: Name
  Product DPr2b
    Spec The system must store the following user data: User type
  Product DPr2c
    Spec The system must store the following user data: Fingerprint
Product DPr3
  Spec The system must store route data
Product DPr4
  Spec The system must store a service log
Product DPr5
  Spec The system must be able to store future traffic laws
Product DPr6
  Spec The system must store Swedish car inspections rules and regulations
Product DPr7
  Spec If there are users in the car, at least one needs to be authenticated
Product DPr8

\end{lstlisting}
    
        
       \section{DesignRequirements}


\begin{lstlisting}
Design De1
  Spec The touch screen must have brightness settings
Design De2
  Spec The touch screen must be easily readable
Design De3
  Spec The emergency break function must be easily accessible from all seats in the car
Design De4
  Spec The emergency break function must be easy to find
Design De5
  Spec The emergency break function must be easy to press
Design De6
  Spec The voice control system must be intuitive, resembling human communication
Design De7
  Spec Changing to manual mode must be easy

\end{lstlisting}
    
        
       \section{QualityRequirements}


\begin{lstlisting}
Domain QDo1
  Spec The system is able to drive until the next connecting road without a GPS-signal
Product QPr1
  Spec The system must query for new traffic laws/regulations at least once a day
Product QPr2
  Spec The system must query for new car inspection rules/regulations online at least once a day
Product QPr3
  Spec System updates must only be performed when the car is parked
Product QPr4
  Spec Traffic and weather data must be fetched every minute when an Internet connection is available
Quality Qu1
  Spec The car always needs to have fuel left in the tank to be able to drive to the nearest gas station when calculating the route the user has put in the system
Quality Qu2
  Spec All traffic laws that apply to the LADA car or general traffic must be stored
Quality Qu3
  Spec The system must respond to a potential accident before the accident is unavoidable
Quality Qu4
  Spec The Internet connection must be fast enough to download the weather and traffic information within 1 minute
Quality Qu5
  Spec When the emergency break is activated the system must start to process the request within 100 ms
Quality Qu6
  Spec The sensors that monitor the road and surroundings must have a margin of error less than 0.01 %
Quality Qu7
  Spec Ecodriving must be performed according to the current definition by Sveriges Trafikskolors Riksförbund
Quality Qu8
  Spec When a command is entered remotely, the system must send a confirmation to the user that sent the command
Quality Qu9
  Spec Any dangerous weather conditions must be notified when a route is selected, before a route is commenced
Quality Qu10
  Spec The service log must store data about date, service type, service provider, mileage and description from every service done on the car
Quality Qu11
  Spec The margin of error for the voice control system must be lower than 1 %
Quality Qu12
  Spec The GPS system must be precise to within 1 meter
Quality Qu13
  Spec The system must query for updates to its maps once a day
Quality Qu14
  Spec The maps must be precise, compared to the reality, to within 3 meters
Quality Qu15
  Spec The dashboard control must be precise to within 0.5 mm from the touch point of the user
Quality Qu16
  Spec The touch screen responsiveness must be instant

\end{lstlisting}
    
        
       \section{DomainRequirements}


\begin{lstlisting}
Domain FDo1
  Spec The system must adhere to Swedish traffic laws as long as an accident is not imminent
Domain FDo2
  Spec If an accident can be avoided, the system may break traffic laws
Domain FDo3
  Spec When the car is in an unsafe state the car must not be able/allowed to drive
Domain FDo4
  Domain FDo4a
    Spec In the situation of an accident, the system must prioritize risks in the following order: Saving as many human lives as possible has top priority
  Domain FDo4b
    Spec In the situation of an accident, the system must prioritize risks in the following order: Protecting humans inside the car is prioritized over humans outside the car
Domain FDo5
  Spec The system requires adequate sensors for monitoring the road and surroundings
Domain FDo6
  Spec The system requires a network connection
Domain FDo7
  Spec The system requires a positioning instrument
Domain FDo8
  Spec The system requires a intoxication measuring instrument
Domain FDo9
  Spec The system requires an authentication sensor
Domain FDo10
  Spec The system requires a voice input device
Domain FDo11
  Spec The system functions without a GPS-signal once a route has been chosen
Domain FDo12
  Spec The system needs a GPS-signal at startup
Domain FDo13
  Spec The system must strive to ecodrive when possible
Domain FDo14
  Spec All user interaction that is available in the system must be possible to enter remotely
Domain FDo15
  Domain FDo15a
    Spec The system must support: permanent user rights
  Domain FDo15b
    Spec The system must support: temporary user rights
Domain DDo1
  Domain DDo1a
    Spec The system must support the following authorized user types: Passenger
  Domain DDo1b
    Spec The system must support the following authorized user types: Driver
  Domain DDo1c
    Spec The system must support the following authorized user types: Owner
  Domain DDo1d
    Spec The system must support the following authorized user types: Admin
Domain DDo2
  Spec A driver must be able to choose & change destination
Domain DDo3
Domain DDo4
  Spec An admin must have the same rights as an owner and the admin must be able to change system settings

\end{lstlisting}

\end{document}

\documentclass{article}
\usepackage[utf8]{inputenc}
\usepackage{titling}
\usepackage{graphicx}
\usepackage{sidecap} % SCfigur
\usepackage[compact,explicit]{titlesec}
\usepackage{hyperref}
\usepackage{todonotes}

\begin{document}

% Raden nedan ser till att nummret på en subsubsection kommer EFTER dess titel.
% Det är nuvarande lösning för att få kraven på formatet "Requirement 6.1.3"
\titleformat{\subsubsection}[runin]{\large\bfseries}{}{0pt}{#1\quad\thesubsubsection}

\begin{titlepage}

\noindent
\textsc{} \\ % The group responsible for the document, such as Development Group
\textsc{}\\ % Names of the authors of the document 
\textsc{\today}\\ % The date
\textsc{}\\ % The document number, eg PUSS124819. See page 22 in the "Projekthandledning"
\textsc{} \\ % The document version, such as 0.1 or 1.0

\vspace{10em}
\begin{flushright}

{\huge Software Requirements Specification}\\ % The document name

{\large Group F} \\
\par\end{flushright}\vskip 0.5em
\end{titlepage}

\newpage

\section*{Document history}

\begin{center}
    \begin{tabular}{ l l l l }
      Version       &      Date             &       Responsible    &   Description     \\ \hline
      
    \end{tabular}
\end{center}

\newpage

\tableofcontents

\newpage

\section{Introduction}
This document describes the requirements for a system developed in the \emph{PUSS}-project course (ETSN05)
at LTH. The system shall handle time-reporting, project groups, different roles within the project group and administration. All functionality of the system is accessed by a user via a web-interface and all data is stored in a MySQL database. It is possible to extend the system to support for example: other project group constellations, other project methodologies and so on.


\section{Reference documents} % change to normal list to look like SVVS?
\begin{enumerate}
\item SRS – Software Requirements Specification: BaseBlockSystem v1.0 - PUSS12002
\item Projekthandledning, v. 2.1
\end{enumerate}

\section{Background and goals}
\subsection{Main goals}

The main goal of the system is to provide an online time-reporting tool. The system should be based on the 'Base Block System', which is a system developed by the Department of Computer Science and
the Software Engineering Research Group (SERG) that provides log-in and log-out functionality on a Tomcat-server. \\
Project members in the system belong to different project groups in which they report their time spent on various tasks. It is also possible for project members to view summations of the time reported.


\subsection{Actors and their objectives}

\begin{figure}[htb]
  \centering
  \includegraphics[width=0.9\textwidth]%
    {userHierarchy.jpg}% picture filename
  \caption{Showing the user hierarchy of the system.}
  \label{hierarchy}
\end{figure}


{\fontsize{11}{11}\selectfont \noindent\textbf{3.2.1 Users}} A graphical representation of the User hierarchy can be found in Figure \ref{hierarchy}. Since every main actor using the system can be seen as a user of some kind, as seen in the figure, the different types of users have been divided as described below. \\
\textbf{3.2.1.1 Administrator} An administrator is, as mentioned above, one specific type of user, as well as a role. An administrator can add and remove users as well as projects groups, and assign a user to be a project manager of a specific group. The main objective of the administrator is thus to be able to manage the administration part of the system easily. \\
\textbf{3.2.1.2 Project Member} A project member is simply a user that is allowed to be active in projects and can either be a 'Project Worker' or 'Project Manager'. \\
\textbf{3.2.1.3 Project Worker} Project workers can be assigned different roles in a project group, depending on what they will be working on during the project. The main objectives of them are to create time reports in the project group that they are a part of as well as look at the time reports they have submitted so that they can be able to track how much time they have spent on different activities.

\textit{Clarification:} Although it is true that project workers are allowed to have different roles in the project, they cannot have the role of project manager, since a project manager will have more privileges compared to the other roles. Thus a user having the role of project manager simply cannot be seen as a project worker. Please see the 'Project manager' section to read more about the role of project manager.
\\
\textbf{3.2.1.4 Project manager} A project manager is a role which a project member can have for a specific project. What differs between this role and the other available roles (except admin) is that by being a project manager the user gains more privileges such as signing time reports or assigning other roles to the users in the same group as the project manager. Moreover the project managers are responsible for additional project information (for example estimates) and should have the ability to show relevant project metrics calculated by the system.
\\

\begin{figure}[htb]
  \centering
  \includegraphics[width=0.9\textwidth]%
    {context_diagram1.png}% picture filename
  \caption{Context diagram showing actors and key components within the system.}
  \label{ContextDiagram}
\end{figure}

\section{Terminology}
\textbf{Duration} - The difference between the start and end time for a time report. \\
\textbf{Type} - States the type of work that has been done on an activity, for example rework after a formal review. All availabile types can be found in Appendix \ref{types}\\
\textbf{Number} - A number defining what activity a user has been spent time on. All different numbers with their corresponding activity can be found in Appendix \ref{numbers}
\textbf{Activity} - An activity is a part related to the project such as the SRS or STLDD which a user may spend time on during the project and thus can specify to have worked on when creating the time report. This is done by entering a number which corresponds to a certain activity. Furthermore every activity is associated with a type. All available types and activities with their corresponding numbers can be found in Table 1 in Appendix \ref{types} and Appendix \ref{numbers}.  \\
\textbf{Roles} - The current roles in the system are: Admin, Project manager, Developer, System architect, Tester and Unspecified (default) \\
\textbf{Time report} - A time report is simply a report containing a date, duration, week number, and type. Moreover it has two states, namely signed and unsigned. \\
\textbf{Project group} - A group of project members all working on the same project. The users may have different roles in the project as explained earlier.  \\
\textbf{Summation} - Simply a time summation of different durations in time reports, for example per user. \\
\textbf{Time report signing} - A project manager is able to sign a time report, which means that it will be locked for further editing. \\
\textbf{Web Session} - Every user that connects to the server and logs in will have an own session. This session will be destroyed when for example the user logs out, or if the user is inactive for more than 20 minutes on the page. \\
\textbf{Administration page} - The page containing all functionality of the administrator. This page is presented to the administrator when he logs in.\\
\textbf{Project manager page} - The page containing all functionality of a project manager. This page is presented to the project manager when he logs in.\\
\textbf{Project worker page} - The page containing all functionality of a project worker. This page is presented to a project worker when he logs in.\\
\textbf{Default role} The default role is 'Unspecified'. \\
\textbf{Predefined group} There is one predefined group in the system, namely 'AdminGroup'. Only the administrator will be a part of this group. The group exists in order to be consistent since all other users in the system will be part of a group, as well as making it more straightforward code-wise. 

\section{Context diagram}

A context diagram of the system can be seen in Figure \ref{ContextDiagram}.

\section{Functional requirements}

The functional requirements will describe what functionality the system needs to meet. This type of requirements is easy to test since they actually are a specific function in the system which can be carried out. Thus they differ from for example "Quality requirements" which is more abstract and a bit harder to specify and test. 

The subsections are divided into different use areas in the system, for example 'Project management' and 'Administration'. 

Finally, since the requirements from the BaseBlockSystem need to be fulfilled as well, the numbering in this document will be a continuation of the numbering from the SRS for the BaseBlockSystem. However the requirements will have the same subsection numbering as in Ref. 1. This means for example that the subsection for 'Login and logout' will remain 6.1, and the subsection for 'Data' will remain 6.2 etc. 

However in order to make this document more readable the requirements will be divided into further subsections compared to in Ref. 1. This means that the requirements in this document will be numbered 6.1.x.x, instead of 6.1.x as in Ref. 1. This means that for example 'Administration' requirements will be divided into subsections covering 'User creation', 'User removal' and so on, so that those requirements will be easier to find when one is looking for them. In order to avoid confusion to the greatest extent possible the numbering in this SRS will always be on the form 6.1.x.x, 6.2.x.x and so on. \\ \\


\newcounter{adminRef}
\setcounter{adminRef}{0}
\addtocounter{adminRef}{1}
\newcounter{adminScen}
\setcounter{adminScen}{0}
\addtocounter{adminScen}{1}


\noindent{\large\textbf{6.0 Inherited requirements}} \\
The following requirements are based on inheritance from the BaseBlockSystem. \\

\noindent\textbf{Requirement 6.0.1.1} All requirements except for 6.3.3, 6.3.4 and 6.3.8 - 6.3.11 found in Ref. 1 should be supported by the system. However, there will be corresponding requirements in this document, so no functionality is lost, but rather expanded.

\subsection{Login and logout}
The following requirements are based on the login and logout functionality of the system. \\ 

{\fontsize{11}{11}\selectfont \noindent\textbf{6.1.10 - Redirection to pages when logging in}} \\
\textbf{Requirement 6.1.10.1} When the administrator logs in, the system should redirect him to the administration page. \\
\textbf{Requirement 6.1.10.2} When a project manager logs in, the system should redirect him to the project manager page. \\
\textbf{Requirement 6.1.10.3} When a project worker logs in, the system should redirect him to the project worker page. \\

{\fontsize{11}{11}\selectfont \noindent\textbf{6.1.11 - Listing when logged in}} \\
\textbf{Requirement 6.1.11.1} A logged in project member should be able to list all other project members who are part of the project member's group.  \\
\noindent\textbf{Requirement 6.1.11.2} A project member should only be able to see his/her own project group and no other groups. \\


\newcounter{dataReq}
\newcounter{dataScen}
\setcounter{dataScen}{1}
\setcounter{dataReq}{0}
\addtocounter{dataReq}{1}

\newpage
\begin{figure}[htb]
  \centering
  \includegraphics[width=0.99\textwidth]%
    {ERdiagram.png}% picture filename
  \caption{ER diagram of the database}
  \label{ER}
\end{figure}

\subsection{Data}
The following requirements are based on the data of the system. Moreover these requirements will be sectioned into subsections where each subsection covers what format on the data a part of the system should have, for example 'Time reports'. \\

{\fontsize{11}{11}\selectfont \noindent\textbf{6.2.5 - Time reports}} \\
\textbf{Requirement 6.2.5.1} The date format presented to the project member should be in the following format YYYY-MM-dd. \\
%\addtocounter{dataReq}{1}
\textbf{Requirement 6.2.5.2} The date format submitted to the server should be in the following format YYYY-MM-dd. \\
%\addtocounter{dataReq}{1}
\textbf{Requirement 6.2.5.3} % New requirement
%\addtocounter{dataReq}{-1}
If the date is submitted in a different way than mentioned in requirement 6.2.5.2, an error message should be displayed stating that "Wrong format on input! Please try again!"  \\
%\addtocounter{dataReq}{2}
\textbf{Requirement 6.2.5.4} The type for a time report should be one of the possible values found in Appendix \ref{types}. \\
%\addtocounter{dataReq}{1}
\textbf{Requirement 6.2.5.5} The duration value for a time report should be a natural number and specified in minutes. \\
%\addtocounter{dataReq}{1}
\textbf{Requirement 6.2.5.6} The number for a time report should be one of the possible values found in Appendix \ref{numbers}. \\

{\fontsize{11}{11}\selectfont \noindent\textbf{6.2.6 - Project group}} \\
\textbf{Requirement 6.2.6.1} The project group names should only consist of alphanumeric characters between 1-20 characters. \\
%\addtocounter{dataReq}{1}
\textbf{Requirement 6.2.6.2} A maximum of 5 project managers per group is allowed. If the administrator tries to add more project managers an error message will be displayed, stating that "Maximum number of managers already added.". \\
\textbf{Requirement 6.2.6.3} Only the administrator is allowed to be part of the predefined group 'AdminGroup' If trying to add another project member to it a message stating "Only administrators are allowed to be a part of this group. Please choose another one and try again" .\\
%\addtocounter{dataReq}{1}

{\fontsize{11}{11}\selectfont \noindent\textbf{6.2.7 - Project member}} \\
\textbf{Requirement 6.2.7.1} The only allowed roles for a project member are:
\begin{itemize}
\item Project manager
\item Developer
\item System architect
\item Tester
\item Unspecified (default)
\end{itemize}.\\

{\fontsize{11}{11}\selectfont \noindent\textbf{6.2.8 - Administrator - Removed due to duplicate}} \\
\textbf{Requirement 6.2.8.1} - \textbf{Removed} - The only allowed role for an administrator is 'Admin'.

%\addtocounter{dataReq}{1}

{\fontsize{11}{11}\selectfont \noindent\textbf{6.2.8 - Time outs and validation}} \\
\textbf{Requirement 6.2.8.1} If the web session times out while editing a time report, all data that has been filled in but not has been submitted is lost.\\
%\addtocounter{dataReq}{1}
\textbf{Requirement 6.2.8.2} All input from the project member must be validated before the project member request is processed.\\
%\addtocounter{dataReq}{1}

{\fontsize{11}{11}\selectfont \noindent\textbf{6.2.9 - Limitations}} \\
\textbf{Requirement 6.2.9.1} The maximum number of time reports in the system are: 4,294,967,295.\footnote{Since every time report will have a unique ID number and this ID number will be an unsigned integer in the database, this will be the highest number of time reports possible due to the limitations of an unsigned integer.} \\
%\addtocounter{dataReq}{1}
\textbf{Requirement 6.2.9.2} The maximum number of users in the system are: 4,294,967,295. \footnote{Since every user will have a unique ID number and this ID number will be an unsigned integer in the database, this will be the highest number of users possible due to the limitations of an unsigned integer.} \\
%\addtocounter{dataReq}{1}
\textbf{Requirement 6.2.9.3} The maximum number of project groups in the system are: 4,294,967,295. \footnote{Since every project group will have a unique ID number and this ID number will be an unsigned integer in the database, this will be the highest number of project groups possible due to the limitations of an unsigned integer.}\\
%\addtocounter{dataReq}{1}

{\fontsize{11}{11}\selectfont \noindent\textbf{6.2.10 - Database}} \\
\textbf{Requirement 6.2.10.1} The data model for storing information is as displayed in Figure \ref{ER}.\\
\addtocounter{dataReq}{1}

{\fontsize{11}{11}\selectfont \noindent\textbf{6.2.11 - Administrator}} \\
\textbf{Requirement 6.2.11.1} The only allowed role for an administrator is 'Admin'.



\subsection{Administration}
The following requirements are based on the administration functionality of the system. This section also refers to scenarios in section 6.8, in order to clarify how the system is intended to be used. Moreover these requirements will be sectioned into subsections where each subsection covers a special type of functionality regarding administration, for example 'User creation'. \\

{\fontsize{11}{11}\selectfont \noindent\textbf{6.3.12 - Inherited requirements}} \\
\noindent\textbf{Requirement 6.3.12.1} All requirements under Section 6.5 should be fulfilled for the administrator as well.\\

{\fontsize{11}{11}\selectfont \noindent\textbf{6.3.13 - General requirements for project members}} \\
\noindent\textbf{Requirement 6.3.13.1} All project members should be listed on the administration page with name, password, group and role. \\
\textbf{Requirement 6.3.13.2} A project member in the system must be part of a project group. \\
\textbf{Requirement 6.3.13.3} A project member can be a member of maximum one project group. \\

{\fontsize{11}{11}\selectfont \noindent\textbf{6.3.14 - Project member creation}} \\
\noindent\textbf{Requirement 6.3.14.1} Adding a project member is restricted to the administrator. \\
\textbf{Requirement 6.3.14.2} A project member in the system must be assigned to a project group directly when creating the project member. \\
\textbf{Requirement 6.3.14.3} A newly created project member that is not a project manager will be assigned the default role. \\
\textbf{Requirement 6.3.14.4} When creating a project member and it succeeds, the system should respond with a success message stating that "Project member was greated successfully!".\\ 
\textbf{Requirement 6.3.14.5} - \textbf{Removed} - Scenario 6.8.1.1 - Successful project member creation - should be supported by the system \\
\textbf{Requirement 6.3.14.6} - \textbf{Removed} - When creating a project member and the information does not fulfill requirements 6.2.1, 6.2.2 and 6.2.3 found in Reference Document 2.1 and 6.2.6, the system should respond with a failure message stating that "Wrong format on input! Please try again" and then ask the administrator to enter a username, password, group and whether the person is a project manager for this group again.\\
\textbf{Requirement 6.3.14.7} - \textbf{Removed} - Scenario 6.8.1.2 - Unsuccessful project member creation due to incorrect format on entered values - should be supported by the system \\
\textbf{Requirement 6.3.14.8} - \textbf{Removed} - When creating a project member and the chosen group does not exist, the system should respond with a failure message stating that "The group does not exist! Please enter a valid group and try again!" and then ask the administrator to enter a username, password, group and whether the person is a project manager for this group again.\\
\textbf{Requirement 6.3.14.9} - \textbf{Removed} - Scenario 6.8.1.3 - Unsuccessful project member creation due to group does not exist - should be supported by the system \\
\textbf{Requirement 6.3.14.10} - \textbf{Removed} - When creating a project member and the username already exists, the system should respond with a failure message stating that "Username does already exist! Please choose another one and try again!" and then ask the administrator to enter a username, password, group and whether the person is a project manager for this group.\\
\textbf{Requirement 6.3.14.11} - \textbf{Removed} - Scenario 6.8.1.4 - Unsuccessful project member creation due to username already exists - should be supported by the system \\
\textbf{Requirement 6.3.14.12} - \textbf{Removed} - When creating a project member and the information does not fulfill requirements 6.2.1, 6.2.2 and 6.2.3 found in Reference Document 2.1 and 6.2.6.1, the system should respond with a failure message stating that "Wrong format on input! Please try again" and then ask the administrator to enter a username, password, group and whether the person is a project manager for this group again.\\
\textbf{Requirement 6.3.14.13} Scenario 6.8.1.6 - Successful project member creation - should be supported by the system \\
\textbf{Requirement 6.3.14.14} - \textbf{Removed} - Scenario 6.8.1.7 - Unsuccessful project member creation due to incorrect format on entered values - should be supported by the system \\
\textbf{Requirement 6.3.14.15} - \textbf{Removed} - Scenario 6.8.1.8 - Unsuccessful project member creation due to group does not exist - should be supported by the system \\
\textbf{Requirement 6.3.14.16} - \textbf{Removed} - Scenario 6.8.1.9 - Unsuccessful project member creation due to username already exists - should be supported by the system \\
\textbf{Requirement 6.3.14.17} When creating a project member and the chosen group does not exist, the system should respond with a failure message stating that "The group does not exist! Please enter a valid group and try again!". \\
\textbf{Requirement 6.3.14.18} When creating a project member and the username already exists, the system should respond with a failure message stating that "Username does already exist! Please choose another one and try again!"\\
\textbf{Requirement 6.3.14.19} When creating a project member and the information does not fulfill requirements 6.2.1, 6.2.2 and 6.2.3 found in Reference Document 2.1 and 6.2.6.1, the system should respond with a failure message stating that "Wrong format on input! Please try again".\\
\textbf{Requirement 6.3.14.20} Scenario 6.8.1.10 - Unsuccessful project member creation due to incorrect format on entered values - should be supported by the system \\
\textbf{Requirement 6.3.14.21} Scenario 6.8.1.11 - Unsuccessful project member creation due to group does not exist - should be supported by the system \\
\textbf{Requirement 6.3.14.22} Scenario 6.8.1.12 - Unsuccessful project member creation due to username already exists - should be supported by the system \\


{\fontsize{11}{11}\selectfont \noindent\textbf{6.3.15 - Project member removal}} \\
\noindent\textbf{Requirement 6.3.15.1} Removing a project member is restricted to the administrator. \\
\textbf{Requirement 6.3.15.2} If a project member that is logged in gets removed by the administrator, the project member will be logged out and directed to the log in page upon interaction with the system.\\
\textbf{Requirement 6.3.15.3} Scenario 6.8.1.5 - project member deleted while logged in to the system - should be supported by the system. \\
\textbf{Requirement 6.3.15.4} When a project member is removed by the administrator, all of his time reports should be kept in the database. \\

{\fontsize{11}{11}\selectfont \noindent\textbf{6.3.16 - Roles administration}} \\
\noindent\textbf{Requirement 6.3.16.1} The administrator should be able to assign a project member the role of project manager although the project member is already created. This will result in the project member becoming a project manager of the group he is a member of. \\
\noindent\textbf{Requirement 6.3.16.2} The administrator should be able to revoke a project member's role of being a project manager. \\
\noindent\textbf{Requirement 6.3.16.3} If the administrator revokes a project member's role of being a project manager, the project member should obtain the default role instead. \\
\noindent\textbf{Requirement 6.3.16.4} Promoting a project member's role to project manager is restricted to the administrator. \\
\noindent\textbf{Requirement 6.3.16.5} Revoking a project member's role of being a project manager is restricted do the administrator. \\
\noindent\textbf{Requirement 6.3.16.6} If a project member that is logged in gets promoted to project manager, the project member will be logged out and directed to the log in page upon interaction with the system.\\
\noindent\textbf{Requirement 6.3.16.7} Scenario 6.8.3.1 - The administrator makes an online project member project manager - should be supported by the system. \\
\textbf{Requirement 6.3.16.8} - \textbf{Removed} - If a project manager gets demoted to project member while logged in, the project member will be logged out and directed to the log in page upon interaction with the system.\\
\textbf{Requirement 6.3.16.9} Scenario 6.8.3.3 - The administrator revokes a project manager B the project manager role while B is logged in - should be supported by the system. \\
\textbf{Requirement 6.3.16.10} If a project manager gets demoted to project worker while logged in, the project member will be logged out and directed to the log in page upon interaction with the system.\\

{\fontsize{11}{11}\selectfont \noindent\textbf{6.3.17 - Project group creation}} \\
\textbf{Requirement 6.3.17.1} When creating a project group and it succeeds, the system should respond with a success message stating that "Project group was created successfully!". \\
\textbf{Requirement 6.3.17.2} - \textbf{Removed} - Scenario 6.8.2.1 - Successful project group creation - should be supported by the system. \\
\textbf{Requirement 6.3.17.3} - \textbf{Removed} - When creating a project group and the project group already exists, the system should respond with a failure message stating that "The project group does already exist! Please enter another project group name and try again" then ask the administrator to enter a project group name again. \\
\textbf{Requirement 6.3.17.4} - \textbf{Removed} - Scenario 6.8.2.2 - Unsuccessful project group creation due to group already exists - should be supported by the system. \\
\textbf{Requirement 6.3.17.5} - \textbf{Removed} - When creating a project group and the name of the group is entered in the wrong format, the system should respond with a failure message stating that "Wrong format on input! Please try again!" then ask the administrator to enter a project group name. \\
\textbf{Requirement 6.3.17.6} - \textbf{Removed} - Scenario 6.8.2.3 - Unsuccessful project group creation due to incorrect format on entered values - should be supported by the system. \\
\textbf{Requirement 6.3.17.7} Scenario 6.8.2.4 - Successful project group creation - should be supported by the system. \\
\textbf{Requirement 6.3.17.8} - \textbf{Removed} - Scenario 6.8.2.5 - Unsuccessful project group creation due to group already exists - should be supported by the system. \\
\textbf{Requirement 6.3.17.9} - \textbf{Removed} - Scenario 6.8.2.6 - Unsuccessful project group creation due to incorrect format on entered values - should be supported by the system. \\
\textbf{Requirement 6.3.17.10} When creating a project group and the project group already exists, the system should respond with a failure message stating that "The project group does already exist! Please enter another project group name and try again". \\
\textbf{Requirement 6.3.17.11} - \textbf{Removed} - Scenario 6.8.2.7 - Unsuccessful project group creation due to group already exists - should be supported by the system. \\
\textbf{Requirement 6.3.17.12} - \textbf{Removed} - Scenario 6.8.2.8 - Unsuccessful project group creation due to incorrect format on entered values - should be supported by the system. \\
\textbf{Requirement 6.3.17.13} When creating a project group and the name of the group is entered in the wrong format, the system should respond with a failure message stating that "Wrong format on input! Please try again!". \\
\textbf{Requirement 6.3.17.14} Scenario 6.8.2.8 - Unsuccessful project group creation due to group already exists - should be supported by the system. \\
\textbf{Requirement 6.3.17.15}  Scenario 6.8.2.7 - Unsuccessful project group creation due to incorrect format on entered values - should be supported by the system. \\

{\fontsize{11}{11}\selectfont \noindent\textbf{6.3.18 - Project group removal}} \\
\textbf{Requirement 6.3.18.1} The administrator should be able to remove a project group in the system. \\
\textbf{Requirement 6.3.18.2} Any deletion of a project group should be preceded by a confirmation dialog “Are you sure you want to delete project group X: Yes / No”. When the administrator selects “Yes”, the project group is deleted and the administrator is returned to an updated list of project groups. When selecting “No”, the administrator is returned to the current page without the project group being removed. \\
\textbf{Requirement 6.3.18.3} When a project group is removed by the administrator, all of its members should be deleted from the system.  \\
\textbf{Requirement 6.3.18.4} When a project group is removed by the administrator, all of its related time reports should be deleted from the system.  \\

{\fontsize{11}{11}\selectfont \noindent\textbf{6.3.19 - Moving project members between project groups}} \\
\noindent\textbf{Requirement 6.3.19.1} The administrator should be able to move a project member to another project group. \\
\textbf{Requirement 6.3.19.2} If a project member that is logged in gets moved to another group, the project member will be logged out and directed to the log in page upon interaction with the system.\\
\textbf{Requirement 6.3.19.3} Scenario 6.8.3.2 - project member moved to another group while logged in to the system - should be supported by the system. \\
\noindent\textbf{Requirement 6.3.19.4} Moving a project member to another group is restricted to the administrator. \\
\addtocounter{adminRef}{1}
\noindent\textbf{Requirement 6.3.19.5} If a project member is moved to another project group, the project member will maintain his role of the old group if nothing else is specified.  \\
\textbf{Requirement 6.3.19.6} If a project member is moved to another group, his/hers old time reports will stay in the old group and not follow the project member to the new one. \\

{\fontsize{11}{11}\selectfont \noindent\textbf{6.3.20 - Editing project members' credentials}} \\
\noindent\textbf{Requirement 6.3.20.1} The administrator should be able to edit an existing project member's username. \\
\noindent\textbf{Requirement 6.3.20.2} The administrator should be able to edit an existing project member's password. \\
\noindent\textbf{Requirement 6.3.20.3} Editing a project member's username is restricted to the administrator. \\
\noindent\textbf{Requirement 6.3.20.4} Editing a project member's password is restricted to the administrator. \\

{\fontsize{11}{11}\selectfont \noindent\textbf{6.3.21 - General requirements}} \\
\textbf{Requirement 6.3.21.1} The administrator should be able to list all of the project groups in the system. \\
\textbf{Requirement 6.3.21.2}  - \textbf{Removed} - The administrator should not be able to be a part of a group.\\
\textbf{Requirement 6.3.21.3} The administrator can only be a part of the predefined group 'AdminGroup'.\\
\textbf{Requirement 6.3.21.4} The administrator will at all times be part of the predefined group 'AdminGroup'.\\

{\fontsize{11}{11}\selectfont \noindent\textbf{6.3.22 - Project Group Editing}} \\
\textbf{Requirement 6.3.22.1} It should not be possible to edit the name of the 'AdminGroup'.\\
\textbf{Requirement 6.3.22.2} It should not be possible to remove the 'AdminGroup' from the system. \\

\newcounter{timeScen}
\setcounter{timeScen}{0}
\addtocounter{timeScen}{1}
\newcounter{timeRef}
\setcounter{timeRef}{0}
\addtocounter{timeRef}{1}
% ------------------------------------------------------------------------------------------------
%
% General requirements
%
\subsection{General requirements}
The following requirements are based on general functionality of the system.\\ 

\noindent\textbf{Requirement 6.4.2.1}
It should be possible for several project members to use the system simultaneously. \\

\subsection{Interface}
The following requirements are based on the interface of the system.\\

{\fontsize{11}{11}\selectfont \noindent\textbf{6.5.1 - The different pages in the system}} \\
\noindent\textbf{Requirement 6.5.1.1} The startpage when not logged in should look like the picture found in Figure \ref{login} in the Appendix. \\
\noindent\textbf{Requirement 6.5.1.2} When trying logging in using invalid credentials the project member should be redirected to a page looking like the one found in Figure \ref{failedLogin} in the Appendix. \\
\noindent\textbf{Requirement 6.5.1.3} The administration startpage should look like in Figure \ref{adminPage} in the Appendix. \\
\noindent\textbf{Requirement 6.5.1.4} The project manager startpage should look like in Figure \ref{pmPage} in the Appendix. \\
\noindent\textbf{Requirement 6.5.1.5} The project worker startpage  should look like in Figure \ref{ordinaryUserPage} in the Appendix. \\
\noindent\textbf{Requirement 6.5.1.6} The dialog for creating a new project group should look like in Figure \ref{createGroup} in the Appendix. \\
\noindent\textbf{Requirement 6.5.1.6} - \textbf{Removed} - The dialog for creating a new project member should look like in Figure \ref{createUser} in the Appendix. \\
\noindent\textbf{Requirement 6.5.1.7} The dialog for creating a new time report should look like in Figure \ref{createTimeReport} in the Appendix. \\
\noindent\textbf{Requirement 6.5.1.8} The dialog for creating a new project member should look like in Figure \ref{createUser} in the Appendix. \\
\noindent\textbf{Requirement 6.5.1.9} The dialog for deleting a project member should look like in Figure \ref{deleteUser} in the Appendix. \\
\noindent\textbf{Requirement 6.5.1.10} The dialog for deleting a project group should look like in Figure \ref{deleteGroup} in the Appendix. \\
\noindent\textbf{Requirement 6.5.1.11} The dialog for deleting a time report should look like in Figure \ref{deleteReport} in the Appendix. \\
\noindent\textbf{Requirement 6.5.1.12} The dialog for editing a project member should look like in Figure \ref{editUser} in the Appendix. \\
\noindent\textbf{Requirement 6.5.1.13} - \textbf{Removed} - The dialog for editing a project group should look like in Figure \ref{editGroup} in the Appendix. \\
\noindent\textbf{Requirement 6.5.1.14} The dialog for editing a time report should look like in Figure \ref{editReport} in the Appendix. \\
% ------------------------------------------------------------------------------------------------
%
% Time reporting
%
\subsection{Time reporting}
The following requirements are based on the time reporting functionality of the system. This section also refers to scenarios in section 6.8, in order to clarify how the system is intended to be used. Moreover these requirements will be sectioned into subsections where each subsection covers a special type of functionality regarding time reporting, for example 'Time report creation'. \\

{\fontsize{11}{11}\selectfont \noindent\textbf{6.6.1 - Creation of time reports}} \\
\noindent\textbf{Requirement 6.6.1.1} When adding a new time report to the system and it succeeds, the system should respond with a success message stating that "Time report was greated successfully!".\\
%\addtocounter{timeRef}{1}
\textbf{Requirement 6.6.1.2} - \textbf{Removed} - Scenario 6.8.4.1 - Successfully adding a new time report to the system - should be supported by the system. \\
%\addtocounter{timeRef}{1}
\textbf{Requirement 6.6.1.3} When adding a new time report to the system and the data is entered in the wrong format, the system should respond with a failure message stating that "Wrong format on input! Please try again!"\\
%\addtocounter{timeRef}{1}
\textbf{Requirement 6.6.1.4} Scenario 6.8.4.2 - Unsuccessfully adding a new time report to the system - should be supported by the system. \\
%\addtocounter{timeRef}{1}
\textbf{Requirement 6.6.1.5} When creating a new time report it should be connected to the project group the project member is a part of automatically, without the project member entering this information. \\
\addtocounter{timeRef}{1}
\textbf{Requirement 6.6.1.6} When creating a new time report it should be connected to the project member who is creating it automatically, without the project member entering this information. \\
\textbf{Requirement 6.6.1.7} Scenario 6.8.4.8 - Successfully adding a new time report to the system - should be supported by the system. \\

{\fontsize{11}{11}\selectfont \noindent\textbf{6.6.2 - Removal of time reports}} \\
\noindent\textbf{Requirement 6.6.2.1} Any deletion of a time report should be preceded by a confirmation dialog “Are you sure you want to delete this time report: Yes / No”. When the project member selects “Yes”, the time report is deleted and the project member is returned to an updated list of time reports. When selecting “No”, the project member is returned to the current page without the time report being removed. \\
%\addtocounter{timeRef}{1}
\textbf{Requirement 6.6.2.2} Scenario 6.8.4.3 - Successfully remove an existing time report - should be supported by the system. \\
%\addtocounter{timeRef}{1}
\textbf{Requirement 6.6.2.3} Scenario 6.8.4.4 - Unsuccessfully remove an existing time report - should be supported by the system. \\
%\addtocounter{timeRef}{1}


%\addtocounter{timeRef}{1}

{\fontsize{11}{11}\selectfont \noindent\textbf{6.6.3 - Edition of time reports}} \\
\noindent
\textbf{Requirement 6.6.3.1} When editing an unsigned time report and the project member submits the data in the correct format, the system should respond with a success message stating that "Time report updated successfully!".\\
%\addtocounter{timeRef}{1}
\textbf{Requirement 6.6.3.2} - \textbf{Removed} - Scenario 6.8.4.5 - Successfully edit an existing unsigned time report - should be supported by the system.\\
%\addtocounter{timeRef}{1}
\textbf{Requirement 6.6.3.3} When editing an unsigned time report and the new data is entered in the wrong format, the system should respond with a failure message stating that "Wrong format on input! Please try again!"\\
%\addtocounter{timeRef}{1}
\textbf{Requirement 6.6.3.4} Scenario 6.8.4.6 - Unsuccessfully edit an existing unsigned time report - should be supported by the system.\\
\textbf{Requirement 6.6.3.5} - Scenario 6.8.4.9 - Successfully edit an existing unsigned time report - should be supported by the system.\\
%\addtocounter{timeRef}{1}

{\fontsize{11}{11}\selectfont \noindent\textbf{6.6.4 - Signing of time reports}} \\
\noindent
\textbf{Requirement 6.6.4.1} A time report that is created should be unsigned until a project manager signs it. \\
\addtocounter{timeRef}{1}
\textbf{Requirement 6.6.4.2} A time report that is edited should be unsigned until a project manager signs it. \\
\addtocounter{timeRef}{1}
\textbf{Requirement 6.6.4.3} A time report that has been signed by a project manager should not be possible to choose for editing. \\
\addtocounter{timeRef}{1}
\textbf{Requirement 6.6.4.4} A time report that has been signed by a project manager should not be possible to choose for removal. \\
\addtocounter{timeRef}{1}
\textbf{Requirement 6.6.4.5} When editing an unsigned time report and the time report becomes signed while the time report is being edited, submitting the data has no effect to the system. \\
\addtocounter{timeRef}{1}
\textbf{Requirement 6.6.4.6} Scenario 6.8.4.7 - Editing an unsigned time report that becomes signed - should be supported by the system. \\

{\fontsize{11}{11}\selectfont \noindent\textbf{6.6.5 - Viewing time reports}} \\
\noindent
\textbf{Requirement 6.6.5.1} A project member should only be able to see his/her own time reports. \\
\textbf{Requirement 6.6.5.2} A project member should be able to see a summation of his/her time reports:  \\
\indent
\textbf{- 6.6.5.2.a} per date.\\
\indent
\textbf{- 6.6.5.2.b} per activity.\\
\indent
\textbf{- 6.6.5.2.c} duration.\\
\indent
\textbf{- 6.6.5.2.d} signed/unsigned.\\
\indent
\textbf{- 6.6.5.2.e} a combination of the aforementioned attributes.\\
%\addtocounter{timeRef}{1}
\textbf{Requirement 6.6.5.3} It should be possible for the project member to sort the time reports in both ascending and descending order \\
\indent 
\textbf{- 6.6.5.3.a} per date.\\
\indent
\textbf{- 6.6.5.3.b} per activity.\\
\indent
\textbf{- 6.6.5.3.c} duration.\\
\indent
\textbf{- 6.6.5.3.d} signed/unsigned.\\
\indent
\textbf{- 6.6.5.3.e} a combination of the aforementioned attributes.\\
%\addtocounter{timeRef}{1}


{\fontsize{11}{11}\selectfont \noindent\textbf{6.6.6 - General time report requirements}} \\
\noindent
\textbf{Requirement 6.6.6.1} Each time report should belong to a specific project group. \\
%\addtocounter{timeRef}{1}
%\addtocounter{timeRef}{1}
\textbf{Requirement 6.6.6.2} - \textbf{Removed} - Each time report should consist of a
\begin{itemize}
\item date
\item duration
\item type
\item group it belongs to
\item project member  it belongs to
\end{itemize}
\textbf{Requirement 6.6.6.3} Each time report should consist of a
\begin{itemize}
\item date
\item duration
\item number
\item type
\item group it belongs to
\item project member  it belongs to
\end{itemize}
%\addtocounter{timeRef}{1}

%\addtocounter{timeRef}{1}




\newcounter{pmReq}
\setcounter{pmReq}{0}
\addtocounter{pmReq}{1}

\subsection{Project management}
The following requirements are based on the time reporting functionality of the system. Moreover these requirements will be sectioned into subsections where each subsection covers a special type of functionality regarding project management, for example 'Time report signing and unsigning'.\\

{\fontsize{11}{11}\selectfont \noindent\textbf{6.7.1 - Listing}} \\
\noindent\textbf{Requirement 6.7.1.1} - \textbf{Removed} - The project manager of a given project group should be able to see all members in that group. \\
\textbf{Requirement 6.7.1.2} The project manager should be able to see all group members' time reports in the project group.\\
\textbf{Requirement 6.7.1.3} The project manager should be able to see a summation of the time reports:  \\
\indent
\textbf{- 6.7.1.3.a} per role.\\
\indent
\textbf{- 6.7.1.3.b} per project member.\\
\indent
\textbf{- 6.7.1.3.c} per week.\\
\indent
\textbf{- 6.7.1.3.d} per activity.\\
\indent
\textbf{- 6.7.1.3.e} per date.\\
\indent
\textbf{- 6.7.1.3.f} signed/unsigned.\\
\indent
\textbf{- 6.7.1.3.g} duration.\\
\indent
\textbf{- 6.7.1.3.h} different combinations of the aforementioned attributes.\\
\textbf{Requirement 6.7.1.4} The project manager should be able to sort the following data from time reports in both ascending and descending order:  \\
\indent
\textbf{- 6.7.1.4.a} per role.\\
\indent
\textbf{- 6.7.1.4.b} per project member.\\
\indent
\textbf{- 6.7.1.4.c} per week.\\
\indent
\textbf{- 6.7.1.4.d} per activity.\\
\indent
\textbf{- 6.7.1.4.e} per date.\\
\indent
\textbf{- 6.7.1.4.f} signed/unsigned.\\
\indent
\textbf{- 6.7.1.4.g} duration.\\
\indent
\textbf{- 6.7.1.4.h} different combinations of the aforementioned attributes.\\

{\fontsize{11}{11}\selectfont \noindent\textbf{6.7.2 - Time report handling}} \\
\textbf{Requirement 6.7.2.1} The project manager should not be able to update time reports belonging to other project members.\\
\textbf{Requirement 6.7.2.2} The project manager should not be able to delete time reports belonging to other project members.\\

{\fontsize{11}{11}\selectfont \noindent\textbf{6.7.3 - Signing of time reports}} \\
\textbf{Requirement 6.7.3.1} The project manager should be able to sign all group members' time reports in the project group.\\
\textbf{Requirement 6.7.3.2} The project manager should be able to unsign all group members' time reports in the project group.\\

{\fontsize{11}{11}\selectfont \noindent\textbf{6.7.4 - Roles administration}} \\
\textbf{Requirement 6.7.4.1} The project manager should be able to assign group members roles in the project.\\
\textbf{Requirement 6.7.4.2} The project manager should be able to change group members roles in the project.\\


\subsection{Scenarios}
The following section deals with scenarios in order to clarify how the system is intended to be used. Although the scenarios are requirements as well, no new functionality is introduced here. Instead this can be seen as a clarifciation of what can be done, and how it is done.\\

\noindent
{\fontsize{11}{11}\selectfont \noindent\textbf{6.8.1 - Project member management}} \\
\noindent\textbf{Scenario 6.8.1.1} - \textbf{Removed} - Successful project member creation  \\
Precondition: Logged in as the administrator and on the administration startpage.
\begin{enumerate}
\item The administrator presses the button for adding a new project member.
\item The system asks for a username, password, group and whether the person is a project manager for this group
\item The administrator enters the information which should fulfill requirements 6.2.1, 6.2.2 and 6.2.3 found in Reference Document 2.1 and 6.10.6, as well as an existing group in the system.
\item The administrator decides whether the project member should be assigned the role of project manager
\item The administrator chooses to submit the information.
\item The system responds with a success message and returns the administrator to an updated version of the previous page.
\end{enumerate}

\noindent\textbf{Scenario 6.8.1.2} - \textbf{Removed} - Unsuccessful project member creation due to incorrect format on entered values \\
Precondition: Logged in as the administrator and on the administration startpage.
\begin{enumerate}
\item The administrator presses the button for adding a new project member.
\item The system asks for a username, password, group and whether the person is a project manager for this group
\item The administrator enters the information which does not fulfill requirements 6.2.1, 6.2.2 and 6.2.3 found in Reference Document 2.1 and 6.10.6.
\item The administrator decides whether the project member should be assigned the role of project manager
\item The administrator chooses to submit the information.
\item The system responds with a failure message and jumps back to step 2.
\end{enumerate}

\noindent\textbf{Scenario 6.8.1.3} - \textbf{Removed} - Unsuccessful project member creation due to group does not exist  \\
Precondition: Logged in as the administrator and on the administration startpage.
\begin{enumerate}
\item The administrator presses the button for adding a new project member.
\item The system asks for a username, password, group and whether the person is a project manager for this group
\item The administrator enters the information which should fulfill requirements 6.2.1, 6.2.2 and 6.2.3 found in Reference Document 2.1 and 6.10.6 as well as an non-existing group in the system.
\item The administrator decides whether the project member should be assigned the role of project manager
\item The administrator chooses to submit the information.
\item The system responds with a failure message and jumps back to step 2.
\end{enumerate}

\noindent\textbf{Scenario 6.8.1.4} - \textbf{Removed} - Unsuccessful project member creation due to username already exists  \\
Precondition: Logged in as the administrator and on the administration startpage.
\begin{enumerate}
\item The administrator presses the button for adding a new project member.
\item The system asks for a username, password, group and whether the person is a project manager for this group
\item The administrator enters the information which should fulfill requirements 6.2.1, 6.2.2 and 6.2.3 found in Reference Document 2.1 and 6.10.6 as well as an existing group in the system. However an already existing username is entered.
\item The administrator decides whether the project member should be assigned the role of project manager
\item The administrator chooses to submit the information.
\item The system responds with a failure message and jumps back to step 2.
\end{enumerate}

\noindent\textbf{Scenario 6.8.1.5} Project member deleted while logged in to the system \\
Precondition: The project member is logged in.
\begin{enumerate}
\item The administrator removes the project member
\item The project member gets redirected to the login page when trying to access the page in any way.\\
\end{enumerate}

\noindent\textbf{Scenario 6.8.1.6} Successful project member creation  \\
Precondition: Logged in as the administrator and on the administration startpage.
\begin{enumerate}
\item The administrator presses the button for adding a new project member.
\item The system asks for a username, password, group and whether the person is a project manager for this group
\item The administrator enters the information which should fulfill requirements 6.2.1, 6.2.2 and 6.2.3 found in Reference Document 2.1 and 6.2.6.1, as well as an existing group in the system.
\item The administrator decides whether the project member should be assigned the role of project manager
\item The administrator chooses to submit the information.
\item The system responds with a success message and returns the administrator to an updated version of the previous page.
\end{enumerate}

\noindent\textbf{Scenario 6.8.1.7} - \textbf{Removed} - Unsuccessful project member creation due to incorrect format on entered values \\
Precondition: Logged in as the administrator and on the administration startpage.
\begin{enumerate}
\item The administrator presses the button for adding a new project member.
\item The system asks for a username, password, group and whether the person is a project manager for this group
\item The administrator enters the information which does not fulfill requirements 6.2.1, 6.2.2 and 6.2.3 found in Reference Document 2.1 and 6.2.6.1.
\item The administrator decides whether the project member should be assigned the role of project manager
\item The administrator chooses to submit the information.
\item The system responds with a failure message and jumps back to step 2.
\end{enumerate}

\noindent\textbf{Scenario 6.8.1.8} - \textbf{Removed} - Unsuccessful project member creation due to group does not exist  \\
Precondition: Logged in as the administrator and on the administration startpage.
\begin{enumerate}
\item The administrator presses the button for adding a new project member.
\item The system asks for a username, password, group and whether the person is a project manager for this group
\item The administrator enters the information which should fulfill requirements 6.2.1, 6.2.2 and 6.2.3 found in Reference Document 2.1 and 6.2.6.1 as well as an non-existing group in the system.
\item The administrator decides whether the project member should be assigned the role of project manager
\item The administrator chooses to submit the information.
\item The system responds with a failure message and jumps back to step 2.
\end{enumerate}

\noindent\textbf{Scenario 6.8.1.9} - \textbf{Removed} - Unsuccessful project member creation due to username already exists  \\
Precondition: Logged in as the administrator and on the administration startpage.
\begin{enumerate}
\item The administrator presses the button for adding a new project member.
\item The system asks for a username, password, group and whether the person is a project manager for this group
\item The administrator enters the information which should fulfill requirements 6.2.1, 6.2.2 and 6.2.3 found in Reference Document 2.1 and 6.2.6.1 as well as an existing group in the system. However an already existing username is entered.
\item The administrator decides whether the project member should be assigned the role of project manager
\item The administrator chooses to submit the information.
\item The system responds with a failure message and jumps back to step 2.
\end{enumerate}

\noindent\textbf{Scenario 6.8.1.10} Unsuccessful project member creation due to incorrect format on entered values \\
Precondition: Logged in as the administrator and on the administration startpage.
\begin{enumerate}
\item The administrator presses the button for adding a new project member.
\item The system asks for a username, password, group and whether the person is a project manager for this group
\item The administrator enters the information which does not fulfill requirements 6.2.1, 6.2.2 and 6.2.3 found in Reference Document 2.1 and 6.2.6.1.
\item The administrator decides whether the project member should be assigned the role of project manager
\item The administrator chooses to submit the information.
\end{enumerate}

\noindent\textbf{Scenario 6.8.1.11} Unsuccessful project member creation due to group does not exist  \\
Precondition: Logged in as the administrator and on the administration startpage.
\begin{enumerate}
\item The administrator presses the button for adding a new project member.
\item The system asks for a username, password, group and whether the person is a project manager for this group
\item The administrator enters the information which should fulfill requirements 6.2.1, 6.2.2 and 6.2.3 found in Reference Document 2.1 and 6.2.6.1 as well as an non-existing group in the system.
\item The administrator decides whether the project member should be assigned the role of project manager
\item The administrator chooses to submit the information.
\end{enumerate}

\noindent\textbf{Scenario 6.8.1.12} Unsuccessful project member creation due to username already exists  \\
Precondition: Logged in as the administrator and on the administration startpage.
\begin{enumerate}
\item The administrator presses the button for adding a new project member.
\item The system asks for a username, password, group and whether the person is a project manager for this group
\item The administrator enters the information which should fulfill requirements 6.2.1, 6.2.2 and 6.2.3 found in Reference Document 2.1 and 6.2.6.1 as well as an existing group in the system. However an already existing username is entered.
\item The administrator decides whether the project member should be assigned the role of project manager
\item The administrator chooses to submit the information.
\end{enumerate}


\noindent
{\fontsize{11}{11}\selectfont \noindent\textbf{6.8.2 - Project group management}} \\
\noindent\textbf{Scenario 6.8.2.1} - \textbf{Removed} - Successful project group creation \\
Precondition: Logged in as the administrator and on the administration startpage.
\begin{enumerate}
\item The administrator presses the button for adding a new project.
\item The system asks for a project group name.
\item The administrator enters a project group name fulfilling requirement 6.10.6, and which does not exist, and confirms his choice by pressing the button to create the project group
\item The system responds with a success message and returns the administrator to an updated version of the previous page.
\end{enumerate}

\noindent\textbf{Scenario 6.8.2.2} - \textbf{Removed} - Unsuccessful project group creation due to group already exists \\
Precondition: Logged in as the administrator and on the administration startpage.
\begin{enumerate}
\item The administrator presses the button for adding a new project.
\item The system asks for a project group name.
\item The administrator enters a project group name fulfilling requirement 6.10.6, but that already exists, and confirms his choice by pressing the button to create the project group
\item The system responds with a failure message and jumps back to step 2.
\end{enumerate}

\noindent\textbf{Scenario 6.8.2.3} - \textbf{Removed} - Unsuccessful project group creation due to incorrect format on entered values\\
Precondition: Logged in as the administrator and on the administration startpage.
\begin{enumerate}
\item The administrator presses the button for adding a new project.
\item The system asks for a project group name.
\item The administrator enters a project group name which does not fulfill requirement 6.10.6 and confirms his choice by pressing the button to create the project group
\item The system responds with a failure message and jumps back to step 2.
\end{enumerate}

\noindent\textbf{Scenario 6.8.2.4} Successful project group creation \\
Precondition: Logged in as the administrator and on the administration startpage.
\begin{enumerate}
\item The administrator presses the button for adding a new project.
\item The system asks for a project group name.
\item The administrator enters a project group name fulfilling requirement 6.2.6.1, and which does not exist, and confirms his choice by pressing the button to create the project group
\item The system responds with a success message and returns the administrator to an updated version of the previous page.
\end{enumerate}

\noindent\textbf{Scenario 6.8.2.5} - \textbf{Removed} - Unsuccessful project group creation due to group already exists \\
Precondition: Logged in as the administrator and on the administration startpage.
\begin{enumerate}
\item The administrator presses the button for adding a new project.
\item The system asks for a project group name.
\item The administrator enters a project group name fulfilling requirement 6.2.6.1, but that already exists, and confirms his choice by pressing the button to create the project group
\item The system responds with a failure message and jumps back to step 2.
\end{enumerate}

\noindent\textbf{Scenario 6.8.2.6} -\textbf{Removed} - Unsuccessful project group creation due to incorrect format on entered values\\
Precondition: Logged in as the administrator and on the administration startpage.
\begin{enumerate}
\item The administrator presses the button for adding a new project.
\item The system asks for a project group name.
\item The administrator enters a project group name which does not fulfill requirement 6.2.6.1 and confirms his choice by pressing the button to create the project group
\item The system responds with a failure message and jumps back to step 2.
\end{enumerate}

\noindent\textbf{Scenario 6.8.2.7} Unsuccessful project group creation due to incorrect format on entered values\\
Precondition: Logged in as the administrator and on the administration startpage.
\begin{enumerate}
\item The administrator presses the button for adding a new project.
\item The system asks for a project group name.
\item The administrator enters a project group name which does not fulfill requirement 6.2.6.1 and confirms his choice by pressing the button to create the project group
\end{enumerate}

\noindent\textbf{Scenario 6.8.2.8}  Unsuccessful project group creation due to group already exists \\
Precondition: Logged in as the administrator and on the administration startpage.
\begin{enumerate}
\item The administrator presses the button for adding a new project.
\item The system asks for a project group name.
\item The administrator enters a project group name fulfilling requirement 6.2.6.1, but that already exists, and confirms his choice by pressing the button to create the project group
\end{enumerate}

\noindent
{\fontsize{11}{11}\selectfont \noindent\textbf{6.8.3 - Administrator actions}} \\
\noindent\textbf{Scenario 6.8.3.1} The administrator makes an online project member project manager \\
Precondition: The administrator is logged in, as well as the project member who is about to be appointed to project manager for a group.
\begin{enumerate}
\item The administrator chooses to edit the project member who should be a project manager

'\item The administrator marks the project member as project manager and submits the changes
\item The project member gets redirected to the login page when trying to access anything on the homepage.
\end{enumerate}

\noindent\textbf{Scenario 6.8.3.2} Project member moved to another group while logged in to the system \\
Precondition: The project member is logged in.
\begin{enumerate}
\item The administrator moves the project member to another group
\item The project member gets redirected to the login page when trying to access anything on the homepage.
\end{enumerate}

\noindent\textbf{Scenario 6.8.3.3} The administrator revokes a project manager B the project manager role while B is logged in.\\
Precondition: The administrator is logged in as well as B. B is currently a project manager.
\begin{enumerate}
\item The administrator chooses to edit the project member B
\item The administrator revokes the project manager role for project member B
\item B gets redirected to login page when trying to access anything on the homepage.\\
\end{enumerate}

\noindent
{\fontsize{11}{11}\selectfont \noindent\textbf{6.8.4 - Time reports}} \\
\noindent\textbf{Scenario 6.8.4.1} - \textbf{Removed} - Successfully adding a new time report to the system \\
Precondition: The project member is logged in and is on the project worker startpage.
\begin{enumerate}
\item The project member presses the indicated button for adding a new time report
\item The project member is presented with a dialog where the person is asked to enter information about the time report
\item The project member enters data into the fields in the dialog presented to him in the correct format as specified in 6.10.1, 6.10.3 and 6.10.4
\item The system responds with a success message.
\end{enumerate}

\noindent\textbf{Scenario 6.8.4.2} Unsuccessfully adding a new time report to the system \\
Precondition: The project member is logged in and is on the project worker startpage.
\begin{enumerate}
\item The project member presses the indicated button for adding a new time report
\item The project member is presented with a dialog where the person is asked to enter information about the time report
\item The project member enters data into the fields in the dialog presented to him, but on incorrect format.
\item The system responds with a failure message.
\end{enumerate}

\noindent\textbf{Scenario 6.8.4.3} Successfully remove an existing time report. \\
Precondition: The project member is logged in and has a previously posted time report.
\begin{enumerate}
\item The project member accesses the system.
\item The project member navigates to the previously posted time reports.
\item The project member selects a specific time report to remove.
\item The system shows a dialog that informs the project member of the removal.
\item The project member confirms the removal by accepting the operation through the dialog.
\end{enumerate}

\noindent\textbf{Scenario 6.8.4.4} Unsuccessfully remove an existing time report. \\
Precondition: The project member is logged in and has a previously posted time report.
\begin{enumerate}
\item The project member accesses the system.
\item The project member navigates to the previously posted time reports.
\item The project member selects a specific time report to remove.
\item The system shows a dialog that informs the project member of the removal.
\item The project member declines the removal by not accepting the operation through the dialog.
\end{enumerate}

\noindent\textbf{Scenario 6.8.4.5} - \textbf{Removed} - Successfully edit an existing unsigned time report. \\
Precondition: The project member is logged in and has previously posted a time report.
\begin{enumerate}
\item The project member navigates to previously posted time reports.
\item The project member selects an unsigned time report of his choice. 
\item The project member selects a specific time report to edit.
\item The project member enters the new input values to the time report in the correct format as specified in Requirement 6.10.1, 6.10.3 and 6.10.4
\item The project member submits the edited information.
\item The system responds with a success message and the page is then refreshed.
\end{enumerate}

\noindent\textbf{Scenario 6.8.4.6} Unsuccessfully edit an existing unsigned time report. \\
Precondition: The project member is logged in and has previously posted a time report.
\begin{enumerate}
\item The project member navigates to previously posted time reports.
\item The project member selects an unsigned time report of his choice. 
\item The project member selects a specific time report to edit.
\item The project member enters the new input values to the time report but in incorrect format.
\item The project member submits the edited information.
\item The system responds with a failure message.
\end{enumerate}

\noindent\textbf{Scenario 6.8.4.7} Editing an unsigned time report that becomes signed \\
Precondition: The project member is logged in and has a previously posted unsigned time report.
\begin{enumerate}
\item The project member accesses the system.
\item The project member edits the unsigned time report
\item The project manager signs the time report
\item The project member submits the data and is returned to the previous page, but will not see the data he submitted since the time report was signed by the project manager
\end{enumerate}

\noindent\textbf{Scenario 6.8.4.8} Successfully adding a new time report to the system \\
Precondition: The project member is logged in and is on the project worker startpage.
\begin{enumerate}
\item The project member presses the indicated button for adding a new time report
\item The project member is presented with a dialog where the person is asked to enter information about the time report
\item The project member enters data into the fields in the dialog presented to him in the correct format as specified in Requirement 6.2.5.2, 6.2.5.4, 6.2.5.5 and 6.2.5.6.
\item The system responds with a success message.
\end{enumerate}

\noindent\textbf{Scenario 6.8.4.9} Successfully edit an existing unsigned time report. \\
Precondition: The project member is logged in and has previously posted a time report.
\begin{enumerate}
\item The project member navigates to previously posted time reports.
\item The project member selects an unsigned time report of his choice. 
\item The project member selects a specific time report to edit.
\item The project member enters the new input values to the time report in the correct format as specified in Requirement 6.2.5.2, 6.2.5.4, 6.2.5.5 and 6.2.5.6
\item The project member submits the edited information.
\item The system responds with a success message and the page is then refreshed.
\end{enumerate}

\section{Quality requirements}

Quality requirements, in contrast to functional requirements, are harder to test since they often are more abstract or may require more work compared to perform a simple function of the system and see if it went as expected, as can be done with functional requirements. However, they still play an important role among requirements since they will help assure that the product actually can be used in practice and not just in theory which the functional requirements focus more on.

\setcounter{subsection}{2}
\subsection{Usability}
The following requirements are based on of usable the system should be.\\

{\fontsize{11}{11}\selectfont \noindent\textbf{7.3.1 - Hands-on}} \\
\noindent\textbf{Requirement 7.3.1.1} 
Four out of five people should find the system easy to use. \\

{\fontsize{11}{11}\selectfont \noindent\textbf{7.3.2 - Browsers}} \\
\textbf{Requirement 7.3.2.1} 
The website shall work in the latest stable version of either Chrome or Firefox.\\

\textbf{Requirement 7.3.2.2} 
The browser needs to support javascript and have it enabled.\\

{\fontsize{11}{11}\selectfont \noindent\textbf{7.3.3 - Monitors}} \\
\textbf{Requirement 7.3.3.1} 
The system is intended to work on monitors having a resolution of at least 800x600. \\

\subsection{Extendability}
The following requirements are based on how extendable the system should be. \\

{\fontsize{11}{11}\selectfont \noindent\textbf{7.4.1 - Scalability}} \\
\noindent\textbf{Requirement 7.4.1.1}
The system should be designed in such a way that it can be expanded to include more types of reports without having to make design changes to the old one.

\subsection{Trustable}
The following requirements are based on how trustable the system should be.\\

{\fontsize{11}{11}\selectfont \noindent\textbf{7.5.1 - General requirements}} \\
\noindent\textbf{Requirement 7.5.1.1}
All changes made to a system should be saved immediately after the project member actively has chosen to make the changes (for example by pressing a button), so that no data is lost in case of system failure. \\

\subsection{Pre-installed software}
The following requirements are based on what software needs to be pre-installed in order to use the system.\\

{\fontsize{11}{11}\selectfont \noindent\textbf{7.6.1 - Software requirements}} \\
\noindent\textbf{Requirement 7.6.1.1} 
Java 7 or newer needs to be installed on the intended server for the system to make it run properly. \\
\textbf{Requirement 7.6.1.2} 
Tomcat 7.0.55 or newer needs to be installed on the intended server for the system to make it run properly. \\
\textbf{Requirement 7.6.1.3}
MySQL 5.5.13 or newer needs to be installed on the intended server for the system to make it run properly. \\

\section{Project requirements}

Project requirements focus more on how the system should be developed rather than how it should work. These requirements are important as well since a customer can make sure of what development environment that is used, or what documentation that will be written. This is useful especially if a company develops the system, and the customer needs another company to continue working on it. If no requirements on documentation, or development environment for example exist, it may be very hard for the new company to take over the project since it may be badly documented, or being developed in internal applications of the first company.
\setcounter{subsection}{1}
\subsection{Development environment}
The following requirements are based on what software the system should be developed in.\\

{\fontsize{11}{11}\selectfont \noindent\textbf{8.2.1 - Storage}} \\
\noindent\textbf{Requirement 8.2.1.1} In addition to Requirement 8.1.3 in Reference Document 2.1 the MySQL database should also be used to store data that should be saved between sessions in this system and not just for the BaseBlockSystem.

\subsection{Documentation}
The following requirements are based on the documentation of the system.\\

{\fontsize{11}{11}\selectfont \noindent\textbf{8.3.1 - Language}} \\
\noindent\textbf{Requirement 8.3.1.1} % used to be either english or swedish
The system as well as the project- and product documentation should be written in English. \\
\textbf{Requirement 8.3.1.2}
The java code should follow the standard described in \url{http://www.geosoft.no/development/javastyle.html} and all variable names should be in English.
\\
\subsection{Release}
The following requirements states what state the system should be in when delivered.\\

{\fontsize{11}{11}\selectfont \noindent\textbf{8.4.1 - State of the system}} \\
\noindent\textbf{Requirement 8.4.1.1} The administrator account should be the only existing account when the system is delivered. \\
\textbf{Requirement 8.4.1.2} No project groups should exist in the system when the system is delivered.\\
\textbf{Requirement 8.4.1.3} No time reports should exist in the system when the system is delivered. \\

\appendix
\section{Numbers} The following numbers are valid in the time report.
\label{numbers}
\begin{table}[htb]
\centering
\begin{tabular}{|c|c|} \hline
    Number        &   Activity    \\ \hline
    11            &   SDP           \\ \hline
    12            &   SRS           \\ \hline
    13            &   SVVS             \\ \hline 
    14            &   STLDD                 \\ \hline
    15            &   SVVI             \\ \hline
    16            &   SDDD                 \\ \hline
    17            &   SVVR                 \\ \hline
    18            &   SSD                 \\ \hline
    19            &   Project final report                 \\ \hline
    21            &   Functional testing               \\ \hline
    22            &   System testing                 \\ \hline
    23            &   Regression testing               \\ \hline
    30            &   Meeting                 \\ \hline
    41            &   Lecture                 \\ \hline
    42            &   Exercise                 \\ \hline
    43            &   Computer exercise                 \\ \hline
    44            &   Outside reading               \\ \hline
    100            &  Miscellaneous                  \\ \hline
\end{tabular}
\caption{Lists the different activities along side with their corresponding number, which is used when doing time reporting in the system in order to specify which activity the project member has spent time on.}
\end{table}
\newpage

\section{Types} The following types are valid in the time report.
\label{types}
\begin{table}[htb]
\centering
\begin{tabular}{|c|c|} \hline
    Type         &   Usage    \\ \hline
    D            &   \textbf{Development} - Should be used when time has been spent on developing the \\
                 &   initial document before the first informal review           \\ \hline
    I            &   \textbf{Informal} - Should be used when time has been spent on preparing for the \\ 
                 &   informal review, or the informal review itself.        \\ \hline
    F            &   \textbf{Formal} - Should be used when time has been spent on preparing for the \\ 
                 &   formal review, or the formal review itself. \\ \hline 
    R            &   \textbf{Rework} - Should be used when time has been spent on rework of the document \\ 
                 &   typically after a formal review \\ \hline
\end{tabular}
\caption{Lists the different types, with a brief description of what they mean}
\end{table}

\newpage

\section{Mockups}

\begin{figure}[htb]
  \centering
  \includegraphics[width=1.1\textwidth]%
    {loginPage.png}% picture filename
  \caption{The design of the login page of the system.}
  \label{login}
\end{figure}

\begin{figure}[htb]
  \centering
  \includegraphics[width=1.1\textwidth]%
    {failedLoginPage.png}% picture filename
  \caption{The design of the failed login page of the system.}
  \label{failedLogin}
\end{figure}

\begin{figure}[htb]
  \centering
  \includegraphics[width=1.1\textwidth]%
    {adminStartpage.jpg}% picture filename
  \caption{The design of the administration startpage of the system.}
  \label{adminPage}
\end{figure}

\begin{figure}[htb]
  \centering
  \includegraphics[width=1.1\textwidth]%
    {projectManagerStartpage.jpg}% picture filename
  \caption{The design of the project manager startpage of the system.}
  \label{pmPage}
\end{figure}

\begin{figure}[htb]
  \centering
  \includegraphics[width=1.1\textwidth]%
    {ordinaryUserStartpage}% picture filename
  \caption{The design of the project worker startpage of the system.}
  \label{ordinaryUserPage}
\end{figure}

\begin{figure}[htb]
  \centering
  \includegraphics[width=1.1\textwidth]%
    {createNewGroupScenario}% picture filename
  \caption{The design of the dialog when creating a group.}
  \label{createGroup}
\end{figure}

\begin{figure}[htb]
  \centering
  \includegraphics[width=1.1\textwidth]%
    {createNewUserScenario}% picture filename
  \caption{The design of the dialog when creating a project member.}
  \label{createUser}
\end{figure}

\begin{figure}[htb]
  \centering
  \includegraphics[width=1.1\textwidth]%
    {createNewTimeReportScenario}% picture filename
  \caption{The design of the dialog when creating a time report}
  \label{createTimeReport}
\end{figure}

\begin{figure}[htb]
  \centering
  \includegraphics[width=1.1\textwidth]%
    {admin_deleteUser}% picture filename
  \caption{The design of the dialog when deleting a project member}
  \label{deleteUser}
\end{figure}

\begin{figure}[htb]
  \centering
  \includegraphics[width=1.1\textwidth]%
    {admin_deleteGroup}% picture filename
  \caption{The design of the dialog when deleting a project group}
  \label{deleteGroup}
\end{figure}

\begin{figure}[htb]
  \centering
  \includegraphics[width=1.1\textwidth]%
    {delete-timereport}% picture filename
  \caption{The design of the dialog when deleting a time report}
  \label{deleteReport}
\end{figure}

\begin{figure}[htb]
  \centering
  \includegraphics[width=1.1\textwidth]%
    {admin_editUser}% picture filename
  \caption{The design of the dialog when editing a project member}
  \label{editUser}
\end{figure}

\begin{figure}[htb]
  \centering
  \includegraphics[width=1.1\textwidth]%
    {admin_editGroup}% picture filename
  \caption{The design of the dialog when editing a project group. Please note that this requirement has been removed and thus this picture should not be seen as a functionality of the system.}
  \label{editGroup}
\end{figure}

\begin{figure}[htb]
  \centering
  \includegraphics[width=1.1\textwidth]%
    {edit-timereport}% picture filename
  \caption{The design of the dialog when editing a time report}
  \label{editReport}
\end{figure}

\end{document}
